%%%%%%%%%%%%%%%%%%%%%%%%%%%%%%%%%%%%%%%%%%%%%%%%%%%%%%%%%%%%
% Theory book for guitarists
%
%%%%%%%%%%%%%%%%%%%%%%%%%%%%%%%%%%%%%%%%%%%%%%%%%%%%%%%%%%%%

%----------------------------------------------------------------------------------------
%	PACKAGES AND DOCUMENT CONFIGURATIONS
%----------------------------------------------------------------------------------------

\documentclass{article}

\usepackage{parskip}
\usepackage[shortlabels]{enumitem}
%\usepackage[version=3]{mhchem} % Package for chemical equation typesetting
%\usepackage{siunitx} % Provides the \SI{}{} and \si{} command for typesetting SI units
\usepackage{graphicx} % Required for the inclusion of images
%\usepackage{natbib} % Required to change bibliography style to APA
\usepackage{amsmath,amssymb} % Required for some math elements 
\usepackage{caption}
\usepackage{subcaption}
\usepackage{hyperref}
\setlength\parindent{0pt} % Removes all indentation from paragraphs
\usepackage{booktabs}

\usepackage{changepage}
\usepackage{multirow}

\usepackage{pgfplots} % Include package for TikZ and PGF plot
\usepackage{anyfontsize} % enable to change the font size manually
\usepackage{makecell}%
\usetikzlibrary{shapes.geometric}
\tikzset{
  dot/.style = {circle, fill, minimum size=#1,
                inner sep=0pt, outer sep=0pt},
  dot/.default = 6pt
}


\renewcommand{\labelenumi}{\alph{enumi}.} % Make numbering in the enumerate environment by letter rather than number (e.g. section 6)

%% Recherche des images dans les répertoires.
\graphicspath{{./figures/}{./dia/}{./gnuplot/}}

\title{ Music theory for guitar nerds  } % Title
\author{ Jean-Hughes \textsc{Fournier L.} } % Author name
\date{\today} % Date for the report

% Plan:
%
% 1. Intervals (Why intervals are consonant or dissonant?, Harmonic series, harmonic entropy, battement)
%      1.1 Harmonic series
%      1.2 Consonance and dissonance
% 2. Scales    (Major scale, minor scales, pattern on fretboard)
% 3. Chords    (Triad, tetrad, pentad)
%  3.1 Chord progressions (Circle of fifths)
% 4. Modes     (Table of modes)
% 5. Arpeggios



%***********************************************************************************************************************************************
\begin{document}

\maketitle % Insert the title, author and date
\newpage
\tableofcontents
\newpage

\begin{itemize}
	\item Gives the recipe not just examples
	\item If you give a man a fish, you feed him for a day. If you teach a man to fish, you feed him for a lifetime
\end{itemize}

%%%%%%%%%%%%%%%%%%%%%%%%%%%%%%%%%%%%%%%%%%%%%%%%%%%%%%%%%%%%%%%%%%%%%%%
\section{Intervals: where do notes come from?}
%%%%%%%%%%%%%%%%%%%%%%%%%%%%%%%%%%%%%%%%%%%%%%%%%%%%%%%%%%%%%%%%%%%%%%%

%Names of the interval comes from the diatonic scale. In C major scale (C-D-E-F-G-A-B-C) that scale G$\#$ is the augmented 5. In C minor scale (C-D-Eb-F-G-Ab-Bb) G$\#$ is the minor sixth.

\subsection{Harmonic series}

% Figure 
\begin{figure}[h!]
	\centering
%	\hspace*{0cm}
	\scalebox{1}{\documentclass{standalone}
\usepackage{graphicx}
\usepackage{tikz}
\usepackage{pgfplots}
\usepgfplotslibrary{groupplots}
\pgfplotsset{compat=newest}

\pgfplotsset{
    every first x axis line/.style={},
    every first y axis line/.style={},
    every first z axis line/.style={},
    every second x axis line/.style={},
    every second y axis line/.style={},
    every second z axis line/.style={},
    first x axis line style/.style={/pgfplots/every first x axis line/.append style={#1}},
    first y axis line style/.style={/pgfplots/every first y axis line/.append style={#1}},
    first z axis line style/.style={/pgfplots/every first z axis line/.append style={#1}},
    second x axis line style/.style={/pgfplots/every second x axis line/.append style={#1}},
    second y axis line style/.style={/pgfplots/every second y axis line/.append style={#1}},
    second z axis line style/.style={/pgfplots/every second z axis line/.append style={#1}}
}

\makeatletter
\def\pgfplots@drawaxis@outerlines@separate@onorientedsurf#1#2{%
    \if2\csname pgfplots@#1axislinesnum\endcsname
        % centered axis lines handled elsewhere.
    \else
    \scope[/pgfplots/every outer #1 axis line,
        #1discont,decoration={pre length=\csname #1disstart\endcsname, post length=\csname #1disend\endcsname}]
        \pgfplots@ifaxisline@B@onorientedsurf@should@be@drawn{0}{%
            \draw [/pgfplots/every first #1 axis line] decorate {
                \pgfextra
                % exchange roles of A <-> B axes:
                \pgfplotspointonorientedsurfaceabsetupfor{#2}{#1}{\pgfplotspointonorientedsurfaceN}%
                \pgfplots@drawgridlines@onorientedsurf@fromto{\csname pgfplots@#2min\endcsname}%
                \endpgfextra 
                };
        }{}%
        \pgfplots@ifaxisline@B@onorientedsurf@should@be@drawn{1}{%
            \draw [/pgfplots/every second #1 axis line] decorate {
                \pgfextra
                % exchange roles of A <-> B axes:
                \pgfplotspointonorientedsurfaceabsetupfor{#2}{#1}{\pgfplotspointonorientedsurfaceN}%
                \pgfplots@drawgridlines@onorientedsurf@fromto{\csname pgfplots@#2max\endcsname}%
                \endpgfextra 
                };
        }{}%
    \endscope
    \fi
}%
\makeatother



\begin{document}
\begin{tikzpicture}

\definecolor{darkgray176}{RGB}{176,176,176}
\definecolor{gray127}{RGB}{127,127,127}
\definecolor{lightgray204}{RGB}{204,204,204}

\begin{axis}[
legend cell align={left},
legend style={
  fill opacity=0.8,
  draw opacity=1,
  text opacity=1,
  at={(0.97,0.03)},
  anchor=south east,
  draw=lightgray204
},
width=  10cm,
height= 7cm,
tick align=outside,
tick pos=left,
%title={The harmonic series },
x grid style={darkgray176},
%xlabel={Fraction of the period},
xmajorgrids,
xmin=0.0, xmax=0.5,
xtick=      {0, 0.071428, 0.083333, 0.1, 0.125, 0.166666, 0.25, 0.5},
xticklabels={},
xtick style={color=white},
y grid style={darkgray176},
%ymajorgrids,
ymin=-1.07496651041122, ymax=1.3,
ytick={-1,-0.5,0,0.5},
yticklabels={},
ytick style={color=white},
separate axis lines,
first x axis line style=darkgray176,
second x axis line style=darkgray176,
first y axis line style=darkgray176,
second y axis line style=darkgray176
]

% Annotations
\node[black] (note) at (0.064 ,1.20) {{\small $7$}};
\draw[->,black, line width=0.5 pt] (0.0,1.06) -- (0.071428,1.06) ;
\node[black] (note) at (0.080 ,1.15) {{\small $6$}};
\draw[->,black, line width=0.5 pt] (0.0,1) -- (0.08333,1) ;
\node[black] (note) at (0.095 ,1.05) {{\small $5$}};
\draw[->,black, line width=0.5 pt] (0.0,0.94) -- (0.1,0.94) ;
\node[black] (note) at (0.12 ,0.98) {{\small $4$}};
\draw[->,black, line width=0.5 pt] (0.0,0.88) -- (0.125,0.88) ; 
\node[black] (note) at (0.16 ,0.92) {{\small $3$}};
\draw[->,black, line width=0.5 pt] (0.0,0.82) -- (0.166666,0.82) ; 
\node[black] (note) at (0.24 ,0.86) {{\small $2$}};
\draw[->,black, line width=0.5 pt] (0.0,0.76) -- (0.25,0.76) ; 
\node[black] (note) at (0.46 ,0.8) {{\small $1$}};
\draw[->,black, line width=0.5 pt] (0.0,0.7) -- (0.5,0.7) ; 

% Fractions
\node[black] (note) at (0.1 ,0.25) {{\footnotesize $\frac{1}{7}$}};
\node[black] (note) at (0.12 ,0.28) {{\footnotesize $\frac{1}{6}$}};
\node[black] (note) at (0.145 ,0.31) {{\footnotesize $\frac{1}{5}$}};
\node[black] (note) at (0.185 ,0.36) {{\footnotesize $\frac{1}{4}$}};
\node[black] (note) at (0.25 ,0.45) {{\footnotesize $\frac{1}{3}$}};
\node[black] (note) at (0.375 ,0.6) {{\footnotesize $\frac{1}{2}$}};
\node[black] (note) at (0.25 ,-0.85) {{\footnotesize fundamental}};

% Nodes
\node at (0.25,0)[circle,fill,inner sep=1pt]{};
\node at (0.166666,0)[circle,fill,inner sep=1pt]{};
\node at (0.125,0)[circle,fill,inner sep=1pt]{};
\node at (0.1,0)[circle,fill,inner sep=1pt]{};
\node at (0.08333,0)[circle,fill,inner sep=1pt]{};
\node at (0.071428,0)[circle,fill,inner sep=1pt]{};

% Vertical line
\draw[color=black!50!white] (0,0) -- (0.5,0);

\addplot [semithick, black]
table {%
0 -0
0.00251256281407035 -0.015786242013637
0.0050251256281407 -0.0315685497648105
0.00753768844221106 -0.047342989971558
0.0100502512562814 -0.0631056313126736
0.0125628140703518 -0.0788525454074765
0.0150753768844221 -0.0945798077948449
0.0175879396984925 -0.110283498911275
0.0201005025125628 -0.125959705067718
0.0226130653266332 -0.141604519424952
0.0251256281407035 -0.157214042967251
0.0276381909547739 -0.1727843854741
0.0301507537688442 -0.188311666489718
0.0326633165829146 -0.203792016290152
0.0351758793969849 -0.219221576847691
0.0376884422110553 -0.234596502792369
0.0402010050251256 -0.249912962370308
0.042713567839196 -0.265167138398679
0.0452261306532663 -0.280355229217014
0.0477386934673367 -0.29547344963467
0.050251256281407 -0.310518031874169
0.0527638190954774 -0.325485226510212
0.0552763819095477 -0.340371303404113
0.0577889447236181 -0.355172552633428
0.0603015075376884 -0.369885285416547
0.0628140703517588 -0.384505835032011
0.0653266331658292 -0.399030557732341
0.0678391959798995 -0.413455833652134
0.0703517587939698 -0.42777806771021
0.0728643216080402 -0.441993690505579
0.0753768844221106 -0.456099159207016
0.0778894472361809 -0.470090958436003
0.0804020100502513 -0.483965601142839
0.0829145728643216 -0.497719629475683
0.085427135678392 -0.511349615642327
0.0879396984924623 -0.524852162764468
0.0904522613065327 -0.538223905724288
0.092964824120603 -0.551461512003107
0.0954773869346734 -0.564561682511918
0.0979899497487437 -0.577521152413589
0.100502512562814 -0.590336691936528
0.103015075376884 -0.603005107179615
0.105527638190955 -0.615523240908179
0.108040201005025 -0.627887973340858
0.110552763819095 -0.640096222927107
0.113065326633166 -0.652144947115187
0.115577889447236 -0.664031143110431
0.118090452261307 -0.675751848623609
0.120603015075377 -0.687304142609184
0.123115577889447 -0.698685145993306
0.125628140703518 -0.709892022391333
0.128140703517588 -0.720921978814716
0.130653266331658 -0.731772266367077
0.133165829145729 -0.742440180929283
0.135678391959799 -0.752923063833377
0.138190954773869 -0.763218302525168
0.14070351758794 -0.773323331215339
0.14321608040201 -0.78323563151889
0.14572864321608 -0.792952733082778
0.148241206030151 -0.802472214201578
0.150753768844221 -0.811791702421021
0.153266331658291 -0.820908875129262
0.155778894472362 -0.829821460135726
0.158291457286432 -0.838527236237377
0.160804020100503 -0.847024033772299
0.163316582914573 -0.855309735160412
0.165829145728643 -0.863382275431223
0.168341708542714 -0.871239642738459
0.170854271356784 -0.87887987886146
0.173366834170854 -0.886301079693209
0.175879396984925 -0.893501395714874
0.178391959798995 -0.900479032456752
0.180904522613065 -0.907232250945481
0.183417085427136 -0.913759368137437
0.185929648241206 -0.920058757338175
0.188442211055276 -0.926128848607841
0.190954773869347 -0.931968129152435
0.193467336683417 -0.937575143700825
0.195979899497487 -0.942948494867437
0.198492462311558 -0.948086843500509
0.201005025125628 -0.952988909015839
0.203517587939699 -0.95765346971593
0.206030150753769 -0.962079363094463
0.208542713567839 -0.966265486126022
0.21105527638191 -0.970210795540986
0.21356783919598 -0.973914308085538
0.21608040201005 -0.977375100766707
0.218592964824121 -0.980592311082404
0.221105527638191 -0.98356513723637
0.223618090452261 -0.986292838338003
0.226130653266332 -0.988774734587003
0.228643216080402 -0.991010207442792
0.231155778894472 -0.99299869977867
0.233668341708543 -0.994739716020657
0.236180904522613 -0.996232822271007
0.238693467336683 -0.997477646416339
0.241206030150754 -0.998473878220379
0.243718592964824 -0.999221269401276
0.246231155778894 -0.999719633693478
0.248743718592965 -0.999968846894156
0.251256281407035 -0.999968846894156
0.253768844221106 -0.999719633693478
0.256281407035176 -0.999221269401276
0.258793969849246 -0.998473878220379
0.261306532663317 -0.997477646416339
0.263819095477387 -0.996232822271007
0.266331658291457 -0.994739716020657
0.268844221105528 -0.99299869977867
0.271356783919598 -0.991010207442792
0.273869346733668 -0.988774734587003
0.276381909547739 -0.986292838338003
0.278894472361809 -0.98356513723637
0.281407035175879 -0.980592311082404
0.28391959798995 -0.977375100766707
0.28643216080402 -0.973914308085538
0.28894472361809 -0.970210795540986
0.291457286432161 -0.966265486126022
0.293969849246231 -0.962079363094463
0.296482412060302 -0.95765346971593
0.298994974874372 -0.952988909015839
0.301507537688442 -0.94808684350051
0.304020100502513 -0.942948494867437
0.306532663316583 -0.937575143700825
0.309045226130653 -0.931968129152435
0.311557788944724 -0.926128848607841
0.314070351758794 -0.920058757338175
0.316582914572864 -0.913759368137437
0.319095477386935 -0.907232250945481
0.321608040201005 -0.900479032456752
0.324120603015075 -0.893501395714874
0.326633165829146 -0.886301079693209
0.329145728643216 -0.878879878861461
0.331658291457286 -0.87123964273846
0.334170854271357 -0.863382275431224
0.336683417085427 -0.855309735160412
0.339195979899497 -0.847024033772299
0.341708542713568 -0.838527236237378
0.344221105527638 -0.829821460135726
0.346733668341709 -0.820908875129263
0.349246231155779 -0.811791702421021
0.351758793969849 -0.802472214201578
0.35427135678392 -0.792952733082779
0.35678391959799 -0.783235631518891
0.35929648241206 -0.773323331215339
0.361809045226131 -0.763218302525169
0.364321608040201 -0.752923063833377
0.366834170854271 -0.742440180929283
0.369346733668342 -0.731772266367077
0.371859296482412 -0.720921978814716
0.374371859296482 -0.709892022391333
0.376884422110553 -0.698685145993307
0.379396984924623 -0.687304142609184
0.381909547738693 -0.675751848623609
0.384422110552764 -0.664031143110431
0.386934673366834 -0.652144947115187
0.389447236180904 -0.640096222927108
0.391959798994975 -0.627887973340859
0.394472361809045 -0.61552324090818
0.396984924623116 -0.603005107179615
0.399497487437186 -0.590336691936529
0.402010050251256 -0.577521152413589
0.404522613065327 -0.564561682511918
0.407035175879397 -0.551461512003108
0.409547738693467 -0.538223905724289
0.412060301507538 -0.524852162764468
0.414572864321608 -0.511349615642327
0.417085427135678 -0.497719629475683
0.419597989949749 -0.483965601142839
0.422110552763819 -0.470090958436003
0.424623115577889 -0.456099159207016
0.42713567839196 -0.441993690505579
0.42964824120603 -0.42777806771021
0.4321608040201 -0.413455833652134
0.434673366834171 -0.399030557732341
0.437185929648241 -0.384505835032011
0.439698492462312 -0.369885285416547
0.442211055276382 -0.355172552633429
0.444723618090452 -0.340371303404113
0.447236180904523 -0.325485226510212
0.449748743718593 -0.310518031874169
0.452261306532663 -0.29547344963467
0.454773869346734 -0.280355229217015
0.457286432160804 -0.265167138398679
0.459798994974874 -0.249912962370309
0.462311557788945 -0.234596502792369
0.464824120603015 -0.219221576847692
0.467336683417085 -0.203792016290152
0.469849246231156 -0.188311666489718
0.472361809045226 -0.1727843854741
0.474874371859296 -0.157214042967251
0.477386934673367 -0.141604519424952
0.479899497487437 -0.125959705067718
0.482412060301508 -0.110283498911275
0.484924623115578 -0.0945798077948453
0.487437185929648 -0.0788525454074766
0.489949748743719 -0.063105631312674
0.492462311557789 -0.0473429899715582
0.494974874371859 -0.0315685497648106
0.49748743718593 -0.0157862420136373
0.5 -1.22464679914735e-16
};
%\addlegendentry{1*w0 (unison)}
\addplot [semithick, black]
table {%
0 -0
0.00251256281407035 -0.0157842748824053
0.0050251256281407 -0.0315528156563368
0.00753768844221106 -0.0472899038974225
0.0100502512562814 -0.0629798525338588
0.0125628140703518 -0.0786070214836254
0.0150753768844221 -0.0941558332448589
0.0175879396984925 -0.109610788423846
0.0201005025125628 -0.124956481185154
0.0226130653266332 -0.140177614608507
0.0251256281407035 -0.155259015937084
0.0276381909547739 -0.170185651702056
0.0301507537688442 -0.184942642708273
0.0326633165829146 -0.19951527886617
0.0351758793969849 -0.213889033855105
0.0376884422110553 -0.228049579603508
0.0402010050251256 -0.241982800571419
0.042713567839196 -0.255674807821163
0.0452261306532663 -0.269111952862144
0.0477386934673367 -0.282280841255959
0.050251256281407 -0.295168345968264
0.0527638190954774 -0.30776162045409
0.0552763819095477 -0.320048111463554
0.0577889447236181 -0.332015571555216
0.0603015075376884 -0.343652071304592
0.0628140703517588 -0.354946011195666
0.0653266331658292 -0.365886133183538
0.0678391959798995 -0.376461531916689
0.0703517587939698 -0.38666166560767
0.0728643216080402 -0.396476366541389
0.0753768844221106 -0.40589585121051
0.0778894472361809 -0.414910730067863
0.0804020100502513 -0.423512016886149
0.0829145728643216 -0.431691137715612
0.085427135678392 -0.43943993943073
0.0879396984924623 -0.446750697857437
0.0904522613065327 -0.453616125472741
0.092964824120603 -0.460029378669087
0.0954773869346734 -0.465984064576217
0.0979899497487437 -0.471474247433719
0.100502512562814 -0.47649445450792
0.103015075376884 -0.481039681547231
0.105527638190955 -0.485105397770493
0.108040201005025 -0.488687550383354
0.110552763819095 -0.491782568618185
0.113065326633166 -0.494387367293501
0.115577889447236 -0.496499349889335
0.118090452261307 -0.498116411135503
0.120603015075377 -0.499236939110189
0.123115577889447 -0.499859816846739
0.125628140703518 -0.499984423447078
0.128140703517588 -0.499610634700638
0.130653266331658 -0.498738823208169
0.133165829145729 -0.497369858010329
0.135678391959799 -0.495505103721396
0.138190954773869 -0.493146419169001
0.14070351758794 -0.490296155541202
0.14321608040201 -0.486957154042769
0.14572864321608 -0.483132743063011
0.148241206030151 -0.478826734857965
0.150753768844221 -0.474043421750255
0.153266331658291 -0.468787571850413
0.155778894472362 -0.463064424303921
0.158291457286432 -0.456879684068718
0.160804020100503 -0.450239516228376
0.163316582914573 -0.443150539846604
0.165829145728643 -0.43561982136923
0.168341708542714 -0.427654867580206
0.170854271356784 -0.419263618118689
0.173366834170854 -0.410454437564631
0.175879396984925 -0.401236107100789
0.178391959798995 -0.391617815759445
0.180904522613065 -0.381609151262584
0.183417085427136 -0.371220090464642
0.185929648241206 -0.360460989407358
0.188442211055276 -0.349342572996653
0.190954773869347 -0.337875924311804
0.193467336683417 -0.326072473557593
0.195979899497487 -0.313943986670429
0.198492462311558 -0.301502553589807
0.201005025125628 -0.288760576206795
0.203517587939699 -0.275730756001554
0.206030150753769 -0.262426081382234
0.208542713567839 -0.248859814737842
0.21105527638191 -0.235045479218002
0.21356783919598 -0.22099684525279
0.21608040201005 -0.206727916826067
0.218592964824121 -0.192252917516005
0.221105527638191 -0.177586276316714
0.223618090452261 -0.162742613255106
0.226130653266332 -0.147736724817335
0.228643216080402 -0.13258356919934
0.231155778894472 -0.117298251396185
0.233668341708543 -0.101896008145076
0.236180904522613 -0.0863921927370499
0.238693467336683 -0.070802259712476
0.241206030150754 -0.0551417494556375
0.243718592964824 -0.0394262727037383
0.246231155778894 -0.0236714949857791
0.248743718592965 -0.00789312100681863
0.251256281407035 0.00789312100681851
0.253768844221106 0.0236714949857788
0.256281407035176 0.039426272703738
0.258793969849246 0.0551417494556372
0.261306532663317 0.0708022597124757
0.263819095477387 0.0863921927370496
0.266331658291457 0.101896008145076
0.268844221105528 0.117298251396184
0.271356783919598 0.132583569199339
0.273869346733668 0.147736724817335
0.276381909547739 0.162742613255106
0.278894472361809 0.177586276316714
0.281407035175879 0.192252917516005
0.28391959798995 0.206727916826067
0.28643216080402 0.220996845252789
0.28894472361809 0.235045479218001
0.291457286432161 0.248859814737841
0.293969849246231 0.262426081382234
0.296482412060302 0.275730756001554
0.298994974874372 0.288760576206794
0.301507537688442 0.301502553589807
0.304020100502513 0.313943986670429
0.306532663316583 0.326072473557593
0.309045226130653 0.337875924311804
0.311557788944724 0.349342572996653
0.314070351758794 0.360460989407358
0.316582914572864 0.371220090464641
0.319095477386935 0.381609151262584
0.321608040201005 0.391617815759445
0.324120603015075 0.401236107100789
0.326633165829146 0.410454437564631
0.329145728643216 0.419263618118688
0.331658291457286 0.427654867580206
0.334170854271357 0.43561982136923
0.336683417085427 0.443150539846604
0.339195979899497 0.450239516228376
0.341708542713568 0.456879684068718
0.344221105527638 0.463064424303921
0.346733668341709 0.468787571850412
0.349246231155779 0.474043421750255
0.351758793969849 0.478826734857965
0.35427135678392 0.483132743063011
0.35678391959799 0.486957154042769
0.35929648241206 0.490296155541202
0.361809045226131 0.493146419169001
0.364321608040201 0.495505103721396
0.366834170854271 0.497369858010328
0.369346733668342 0.498738823208169
0.371859296482412 0.499610634700638
0.374371859296482 0.499984423447078
0.376884422110553 0.499859816846739
0.379396984924623 0.499236939110189
0.381909547738693 0.498116411135503
0.384422110552764 0.496499349889335
0.386934673366834 0.494387367293501
0.389447236180904 0.491782568618185
0.391959798994975 0.488687550383354
0.394472361809045 0.485105397770493
0.396984924623116 0.481039681547232
0.399497487437186 0.47649445450792
0.402010050251256 0.471474247433719
0.404522613065327 0.465984064576218
0.407035175879397 0.460029378669087
0.409547738693467 0.453616125472741
0.412060301507538 0.446750697857437
0.414572864321608 0.43943993943073
0.417085427135678 0.431691137715612
0.419597989949749 0.423512016886149
0.422110552763819 0.414910730067863
0.424623115577889 0.405895851210511
0.42713567839196 0.396476366541389
0.42964824120603 0.38666166560767
0.4321608040201 0.376461531916689
0.434673366834171 0.365886133183539
0.437185929648241 0.354946011195667
0.439698492462312 0.343652071304592
0.442211055276382 0.332015571555216
0.444723618090452 0.320048111463554
0.447236180904523 0.30776162045409
0.449748743718593 0.295168345968264
0.452261306532663 0.282280841255959
0.454773869346734 0.269111952862145
0.457286432160804 0.255674807821164
0.459798994974874 0.24198280057142
0.462311557788945 0.228049579603508
0.464824120603015 0.213889033855105
0.467336683417085 0.199515278866171
0.469849246231156 0.184942642708274
0.472361809045226 0.170185651702057
0.474874371859296 0.155259015937085
0.477386934673367 0.140177614608508
0.479899497487437 0.124956481185154
0.482412060301508 0.109610788423846
0.484924623115578 0.0941558332448593
0.487437185929648 0.0786070214836255
0.489949748743719 0.0629798525338592
0.492462311557789 0.0472899038974227
0.494974874371859 0.0315528156563369
0.49748743718593 0.0157842748824056
0.5 1.22464679914735e-16
};
%\addlegendentry{2*w0 (octave)}
\addplot [semithick, black]
table {%
0 -0
0.00251256281407035 -0.015780996657186
0.0050251256281407 -0.0315266025982816
0.00753768844221106 -0.0472015064749839
0.0100502512562814 -0.0627705554965726
0.0125628140703518 -0.0781988342641229
0.0150753768844221 -0.0934517430723381
0.0175879396984925 -0.108495075503404
0.0201005025125628 -0.123295095138849
0.0226130653266332 -0.137818611217378
0.0251256281407035 -0.152033053069005
0.0276381909547739 -0.165906543158561
0.0301507537688442 -0.179407968574763
0.0326633165829146 -0.19250705080453
0.0351758793969849 -0.20517441363606
0.0376884422110553 -0.217381649038396
0.0402010050251256 -0.229101380869728
0.042713567839196 -0.240307326271572
0.0452261306532663 -0.250974354611126
0.0477386934673367 -0.26107854383963
0.050251256281407 -0.27059723414034
0.0527638190954774 -0.279509078745792
0.0552763819095477 -0.287794091810408
0.0577889447236181 -0.295433693231069
0.0603015075376884 -0.30241075031516
0.0628140703517588 -0.308709616202614
0.0653266331658292 -0.314316164955812
0.0678391959798995 -0.319217823238643
0.0703517587939698 -0.323403598513662
0.0728643216080402 -0.326864103694135
0.0753768844221106 -0.329591578195668
0.0778894472361809 -0.331579905340219
0.0804020100502513 -0.33282462607346
0.0829145728643216 -0.333322948964719
0.085427135678392 -0.333073756467092
0.0879396984924623 -0.332077607423669
0.0904522613065327 -0.330336735814264
0.092964824120603 -0.327855045745457
0.0954773869346734 -0.324638102695179
0.0979899497487437 -0.320693121031488
0.100502512562814 -0.316028947833503
0.103015075376884 -0.310656043050812
0.105527638190955 -0.304586456045812
0.108040201005025 -0.297833798571625
0.110552763819095 -0.290413214246153
0.113065326633166 -0.282341344590766
0.115577889447236 -0.273636291709754
0.118090452261307 -0.26431757769426
0.120603015075377 -0.254406100841723
0.123115577889447 -0.243924088789026
0.125628140703518 -0.232895048664436
0.128140703517588 -0.221343714370144
0.130653266331658 -0.20929599111362
0.133165829145729 -0.196778897312176
0.135678391959799 -0.183820504001036
0.138190954773869 -0.170449871880776
0.14070351758794 -0.156696986145334
0.14321608040201 -0.142592689236737
0.14572864321608 -0.128168611677337
0.148241206030151 -0.113457101134704
0.150753768844221 -0.0984911498782235
0.153266331658291 -0.0833043207901029
0.155778894472362 -0.0679306720967176
0.158291457286432 -0.0524046809890836
0.160804020100503 -0.0367611663037585
0.163316582914573 -0.021035210437558
0.165829145728643 -0.00526208067121227
0.168341708542714 0.0105228499216034
0.170854271356784 0.026284181802492
0.173366834170854 0.0419865683559057
0.175879396984925 0.0575947951580332
0.178391959798995 0.0730738589492304
0.180904522613065 0.0883890461328927
0.183417085427136 0.103506010624723
0.185929648241206 0.118390850877809
0.188442211055276 0.13301018591078
0.190954773869347 0.147331230168526
0.193467336683417 0.161321867047613
0.195979899497487 0.174950720921489
0.198492462311558 0.188187227503973
0.201005025125628 0.201001702393205
0.203517587939699 0.213365407642369
0.206030150753769 0.22525061620787
0.208542713567839 0.236630674130444
0.21105527638191 0.247480060309761
0.21356783919598 0.257774443738446
0.21608040201005 0.267490738067192
0.218592964824121 0.276607153378575
0.221105527638191 0.285103245053471
0.223618090452261 0.292959959620487
0.226130653266332 0.300159677485584
0.228643216080402 0.306686252446058
0.231155778894472 0.312525047900275
0.233668341708543 0.317662969671946
0.236180904522613 0.322088495375341
0.238693467336683 0.325791700255569
0.241206030150754 0.328764279446001
0.243718592964824 0.33099956659289
0.246231155778894 0.332492548805446
0.248743718592965 0.333239877897826
0.251256281407035 0.333239877897826
0.253768844221106 0.332492548805446
0.256281407035176 0.33099956659289
0.258793969849246 0.328764279446001
0.261306532663317 0.325791700255569
0.263819095477387 0.322088495375341
0.266331658291457 0.317662969671946
0.268844221105528 0.312525047900275
0.271356783919598 0.306686252446058
0.273869346733668 0.300159677485584
0.276381909547739 0.292959959620487
0.278894472361809 0.285103245053471
0.281407035175879 0.276607153378575
0.28391959798995 0.267490738067193
0.28643216080402 0.257774443738446
0.28894472361809 0.247480060309761
0.291457286432161 0.236630674130445
0.293969849246231 0.22525061620787
0.296482412060302 0.213365407642369
0.298994974874372 0.201001702393205
0.301507537688442 0.188187227503973
0.304020100502513 0.174950720921489
0.306532663316583 0.161321867047613
0.309045226130653 0.147331230168527
0.311557788944724 0.133010185910781
0.314070351758794 0.11839085087781
0.316582914572864 0.103506010624723
0.319095477386935 0.0883890461328932
0.321608040201005 0.0730738589492308
0.324120603015075 0.0575947951580332
0.326633165829146 0.0419865683559061
0.329145728643216 0.0262841818024924
0.331658291457286 0.0105228499216034
0.334170854271357 -0.00526208067121215
0.336683417085427 -0.0210352104375577
0.339195979899497 -0.0367611663037579
0.341708542713568 -0.0524046809890832
0.344221105527638 -0.0679306720967173
0.346733668341709 -0.0833043207901025
0.349246231155779 -0.0984911498782231
0.351758793969849 -0.113457101134704
0.35427135678392 -0.128168611677337
0.35678391959799 -0.142592689236736
0.35929648241206 -0.156696986145334
0.361809045226131 -0.170449871880775
0.364321608040201 -0.183820504001036
0.366834170854271 -0.196778897312176
0.369346733668342 -0.209295991113619
0.371859296482412 -0.221343714370144
0.374371859296482 -0.232895048664435
0.376884422110553 -0.243924088789025
0.379396984924623 -0.254406100841723
0.381909547738693 -0.264317577694259
0.384422110552764 -0.273636291709754
0.386934673366834 -0.282341344590766
0.389447236180904 -0.290413214246153
0.391959798994975 -0.297833798571625
0.394472361809045 -0.304586456045812
0.396984924623116 -0.310656043050812
0.399497487437186 -0.316028947833503
0.402010050251256 -0.320693121031487
0.404522613065327 -0.324638102695179
0.407035175879397 -0.327855045745457
0.409547738693467 -0.330336735814264
0.412060301507538 -0.332077607423669
0.414572864321608 -0.333073756467092
0.417085427135678 -0.333322948964719
0.419597989949749 -0.33282462607346
0.422110552763819 -0.331579905340219
0.424623115577889 -0.329591578195668
0.42713567839196 -0.326864103694135
0.42964824120603 -0.323403598513662
0.4321608040201 -0.319217823238643
0.434673366834171 -0.314316164955812
0.437185929648241 -0.308709616202614
0.439698492462312 -0.302410750315161
0.442211055276382 -0.29543369323107
0.444723618090452 -0.287794091810408
0.447236180904523 -0.279509078745793
0.449748743718593 -0.27059723414034
0.452261306532663 -0.26107854383963
0.454773869346734 -0.250974354611126
0.457286432160804 -0.240307326271572
0.459798994974874 -0.229101380869728
0.462311557788945 -0.217381649038396
0.464824120603015 -0.20517441363606
0.467336683417085 -0.192507050804529
0.469849246231156 -0.179407968574763
0.472361809045226 -0.165906543158561
0.474874371859296 -0.152033053069006
0.477386934673367 -0.137818611217378
0.479899497487437 -0.123295095138849
0.482412060301508 -0.108495075503404
0.484924623115578 -0.0934517430723384
0.487437185929648 -0.0781988342641232
0.489949748743719 -0.0627705554965729
0.492462311557789 -0.0472015064749841
0.494974874371859 -0.0315266025982818
0.49748743718593 -0.0157809966571861
0.5 -1.22464679914735e-16
};
%\addlegendentry{3*w0 (P5)}
\addplot [semithick, black]
table {%
0 -0
0.00251256281407035 -0.0157764078281684
0.0050251256281407 -0.0314899262669294
0.00753768844221106 -0.0470779166224295
0.0100502512562814 -0.0624782405925771
0.0125628140703518 -0.0776295079685422
0.0150753768844221 -0.0924713213541367
0.0175879396984925 -0.106944516927552
0.0201005025125628 -0.12099140028571
0.0226130653266332 -0.134555976431072
0.0251256281407035 -0.147584172984132
0.0276381909547739 -0.160024055731777
0.0301507537688442 -0.171826035652296
0.0326633165829146 -0.182943066591769
0.0351758793969849 -0.193330832803835
0.0376884422110553 -0.202947925605255
0.0402010050251256 -0.211756008443075
0.042713567839196 -0.219719969715365
0.0452261306532663 -0.22680806273637
0.0477386934673367 -0.232992032288109
0.050251256281407 -0.23824722725396
0.0527638190954774 -0.242552698885247
0.0552763819095477 -0.245891284309092
0.0577889447236181 -0.248249674944667
0.0603015075376884 -0.249618469555095
0.0628140703517588 -0.249992211723539
0.0653266331658292 -0.249369411604085
0.0678391959798995 -0.247752551860698
0.0703517587939698 -0.245148077770601
0.0728643216080402 -0.241566371531505
0.0753768844221106 -0.237021710875127
0.0778894472361809 -0.23153221215196
0.0804020100502513 -0.225119758114188
0.0829145728643216 -0.217809910684615
0.085427135678392 -0.209631809059344
0.0879396984924623 -0.200618053550395
0.0904522613065327 -0.190804575631292
0.092964824120603 -0.180230494703679
0.0954773869346734 -0.168937962155902
0.0979899497487437 -0.156971993335215
0.100502512562814 -0.144380288103397
0.103015075376884 -0.131213040691117
0.105527638190955 -0.117522739609001
0.108040201005025 -0.103363958413034
0.110552763819095 -0.0887931381583572
0.113065326633166 -0.0738683624086676
0.115577889447236 -0.0586491256980923
0.118090452261307 -0.043196096368525
0.120603015075377 -0.0275708747278188
0.123115577889447 -0.0118357474928896
0.125628140703518 0.00394656050340926
0.128140703517588 0.019713136351869
0.130653266331658 0.0354011298562378
0.133165829145729 0.0509480040725379
0.135678391959799 0.0662917845996696
0.138190954773869 0.081371306627553
0.14070351758794 0.0961264587580027
0.14321608040201 0.110498422626395
0.14572864321608 0.124429907368921
0.148241206030151 0.137865378000777
0.150753768844221 0.150751276794904
0.153266331658291 0.163036236778797
0.155778894472362 0.174671286498327
0.158291457286432 0.185610045232321
0.160804020100503 0.195808907879723
0.163316582914573 0.205227218782316
0.165829145728643 0.213827433790103
0.168341708542714 0.221575269923302
0.170854271356784 0.228439842034359
0.173366834170854 0.234393785925206
0.175879396984925 0.239413367428982
0.178391959798995 0.243478577021384
0.180904522613065 0.246573209584501
0.183417085427136 0.248684929005164
0.185929648241206 0.249805317350319
0.188442211055276 0.249929908423369
0.190954773869347 0.249058205567752
0.193467336683417 0.247193683646751
0.195979899497487 0.244343775191677
0.198492462311558 0.240519840773616
0.201005025125628 0.235737123716859
0.203517587939699 0.230014689334544
0.206030150753769 0.223375348928719
0.208542713567839 0.215845568857806
0.21105527638191 0.207455365033932
0.21356783919598 0.198238183270695
0.21608040201005 0.188230765958344
0.218592964824121 0.177473005597833
0.221105527638191 0.166007785777608
0.223618090452261 0.153880810227045
0.226130653266332 0.14114042062798
0.228643216080402 0.127837403910582
0.231155778894472 0.114024789801754
0.233668341708543 0.0997576394330854
0.236180904522613 0.0850928258510284
0.238693467336683 0.0700888073042538
0.241206030150754 0.0548053942119229
0.243718592964824 0.0393035107418128
0.246231155778894 0.0236449519487113
0.248743718592965 0.00789213744120278
0.251256281407035 -0.00789213744120266
0.253768844221106 -0.023644951948711
0.256281407035176 -0.0393035107418124
0.258793969849246 -0.0548053942119225
0.261306532663317 -0.0700888073042535
0.263819095477387 -0.0850928258510281
0.266331658291457 -0.099757639433085
0.268844221105528 -0.114024789801754
0.271356783919598 -0.127837403910582
0.273869346733668 -0.141140420627979
0.276381909547739 -0.153880810227045
0.278894472361809 -0.166007785777608
0.281407035175879 -0.177473005597833
0.28391959798995 -0.188230765958344
0.28643216080402 -0.198238183270694
0.28894472361809 -0.207455365033931
0.291457286432161 -0.215845568857806
0.293969849246231 -0.223375348928718
0.296482412060302 -0.230014689334544
0.298994974874372 -0.235737123716859
0.301507537688442 -0.240519840773616
0.304020100502513 -0.244343775191677
0.306532663316583 -0.247193683646751
0.309045226130653 -0.249058205567752
0.311557788944724 -0.249929908423369
0.314070351758794 -0.249805317350319
0.316582914572864 -0.248684929005164
0.319095477386935 -0.246573209584501
0.321608040201005 -0.243478577021384
0.324120603015075 -0.239413367428982
0.326633165829146 -0.234393785925206
0.329145728643216 -0.228439842034359
0.331658291457286 -0.221575269923302
0.334170854271357 -0.213827433790103
0.336683417085427 -0.205227218782316
0.339195979899497 -0.195808907879723
0.341708542713568 -0.185610045232321
0.344221105527638 -0.174671286498327
0.346733668341709 -0.163036236778797
0.349246231155779 -0.150751276794904
0.351758793969849 -0.137865378000777
0.35427135678392 -0.124429907368921
0.35678391959799 -0.110498422626395
0.35929648241206 -0.0961264587580031
0.361809045226131 -0.0813713066275532
0.364321608040201 -0.0662917845996698
0.366834170854271 -0.0509480040725384
0.369346733668342 -0.0354011298562381
0.371859296482412 -0.0197131363518691
0.374371859296482 -0.00394656050340949
0.376884422110553 0.0118357474928894
0.379396984924623 0.0275708747278189
0.381909547738693 0.0431960963685247
0.384422110552764 0.0586491256980922
0.386934673366834 0.0738683624086673
0.389447236180904 0.0887931381583566
0.391959798994975 0.103363958413033
0.394472361809045 0.117522739609
0.396984924623116 0.131213040691117
0.399497487437186 0.144380288103397
0.402010050251256 0.156971993335214
0.404522613065327 0.168937962155902
0.407035175879397 0.180230494703679
0.409547738693467 0.190804575631292
0.412060301507538 0.200618053550394
0.414572864321608 0.209631809059344
0.417085427135678 0.217809910684615
0.419597989949749 0.225119758114188
0.422110552763819 0.23153221215196
0.424623115577889 0.237021710875127
0.42713567839196 0.241566371531505
0.42964824120603 0.245148077770601
0.4321608040201 0.247752551860698
0.434673366834171 0.249369411604085
0.437185929648241 0.249992211723539
0.439698492462312 0.249618469555095
0.442211055276382 0.248249674944667
0.444723618090452 0.245891284309092
0.447236180904523 0.242552698885247
0.449748743718593 0.23824722725396
0.452261306532663 0.232992032288109
0.454773869346734 0.22680806273637
0.457286432160804 0.219719969715365
0.459798994974874 0.211756008443075
0.462311557788945 0.202947925605256
0.464824120603015 0.193330832803835
0.467336683417085 0.182943066591769
0.469849246231156 0.171826035652296
0.472361809045226 0.160024055731777
0.474874371859296 0.147584172984132
0.477386934673367 0.134555976431072
0.479899497487437 0.12099140028571
0.482412060301508 0.106944516927552
0.484924623115578 0.092471321354137
0.487437185929648 0.0776295079685423
0.489949748743719 0.0624782405925775
0.492462311557789 0.0470779166224297
0.494974874371859 0.0314899262669294
0.49748743718593 0.0157764078281687
0.5 1.22464679914735e-16
};
%\addlegendentry{4*w0 (2x octave)}
\addplot [semithick, black]
table {%
0 -0
0.00251256281407035 -0.0157705090814953
0.0050251256281407 -0.0314428085934502
0.00753768844221106 -0.0469193005584737
0.0100502512562814 -0.0621036063748338
0.0125628140703518 -0.0769011670064022
0.0150753768844221 -0.0912198318414032
0.0175879396984925 -0.104970432552894
0.0201005025125628 -0.118067338387306
0.0226130653266332 -0.130428989423037
0.0251256281407035 -0.141978404478267
0.0276381909547739 -0.152643660505034
0.0301507537688442 -0.162358340484204
0.0326633165829146 -0.171061947032082
0.0351758793969849 -0.178700279142975
0.0376884422110553 -0.185225769721568
0.0402010050251256 -0.190597781803168
0.042713567839196 -0.194782861617108
0.0452261306532663 -0.197754946917401
0.0477386934673367 -0.199495529283268
0.050251256281407 -0.199993769378831
0.0527638190954774 -0.199246564454201
0.0552763819095477 -0.197258567667601
0.0577889447236181 -0.194042159108197
0.0603015075376884 -0.189617368700102
0.0628140703517588 -0.184011751467635
0.0653266331658292 -0.177260215938642
0.0678391959798995 -0.16940480675446
0.0703517587939698 -0.160494442840316
0.0728643216080402 -0.150584612766675
0.0753768844221106 -0.139737029198661
0.0778894472361809 -0.128019244585422
0.0804020100502513 -0.115504230482718
0.0829145728643216 -0.102269923128465
0.085427135678392 -0.0883987381011159
0.0879396984924623 -0.0739770570833094
0.0904522613065327 -0.0590946899269341
0.092964824120603 -0.0438443153695383
0.0954773869346734 -0.0283209038849904
0.0979899497487437 -0.0126211262625348
0.100502512562814 0.00315724840272732
0.103015075376884 0.0189159615589689
0.105527638190955 0.0345568770948198
0.108040201005025 0.0499825924740616
0.110552763819095 0.0650970453020423
0.113065326633166 0.0798061115464681
0.115577889447236 0.0940181916872005
0.118090452261307 0.107644781144858
0.120603015075377 0.120601021435923
0.123115577889447 0.132806228622086
0.125628140703518 0.144184395762943
0.128140703517588 0.154664666243068
0.130653266331658 0.164181775025852
0.133165829145729 0.172676455086245
0.135678391959799 0.18009580649135
0.138190954773869 0.186393625830487
0.14070351758794 0.191530693943186
0.14321608040201 0.195475020153341
0.14572864321608 0.198202041488558
0.148241206030151 0.199694775644076
0.150753768844221 0.199943926738696
0.153266331658291 0.198947943204131
0.155778894472362 0.196713027447274
0.158291457286432 0.193253097225204
0.160804020100503 0.188589698973487
0.163316582914573 0.182751873627487
0.165829145728643 0.175775975772292
0.168341708542714 0.167705447247476
0.170854271356784 0.158590546616556
0.173366834170854 0.148488036185857
0.175879396984925 0.137460828521837
0.178391959798995 0.125577594668172
0.180904522613065 0.112912336502384
0.183417085427136 0.0995439258951368
0.185929648241206 0.0855556135420419
0.188442211055276 0.0710345105266857
0.190954773869347 0.056071045843403
0.193467336683417 0.0407584032580307
0.195979899497487 0.0251919410135437
0.198492462311558 0.00946859799431166
0.201005025125628 -0.00631370995296195
0.203517587939699 -0.0220566997822549
0.206030150753769 -0.0376623332979434
0.208542713567839 -0.0530334276797357
0.21105527638191 -0.0680742606808224
0.21356783919598 -0.0826911667304268
0.21608040201005 -0.0967931202285676
0.218592964824121 -0.110292302400621
0.221105527638191 -0.123104648181636
0.223618090452261 -0.135150369724722
0.226130653266332 -0.146354453273415
0.228643216080402 -0.156647126303778
0.231155778894472 -0.165964292027145
0.233668341708543 -0.174247928547692
0.236180904522613 -0.181446450189096
0.238693467336683 -0.187515028740165
0.241206030150754 -0.192415872618893
0.243718592964824 -0.196118462216481
0.246231155778894 -0.198599739955734
0.248743718592965 -0.199844253880255
0.251256281407035 -0.199844253880255
0.253768844221106 -0.198599739955734
0.256281407035176 -0.196118462216481
0.258793969849246 -0.192415872618893
0.261306532663317 -0.187515028740165
0.263819095477387 -0.181446450189096
0.266331658291457 -0.174247928547692
0.268844221105528 -0.165964292027145
0.271356783919598 -0.156647126303778
0.273869346733668 -0.146354453273415
0.276381909547739 -0.135150369724722
0.278894472361809 -0.123104648181636
0.281407035175879 -0.110292302400622
0.28391959798995 -0.0967931202285679
0.28643216080402 -0.082691166730427
0.28894472361809 -0.0680742606808225
0.291457286432161 -0.0530334276797362
0.293969849246231 -0.0376623332979437
0.296482412060302 -0.0220566997822552
0.298994974874372 -0.00631370995296243
0.301507537688442 0.00946859799431155
0.304020100502513 0.0251919410135434
0.306532663316583 0.0407584032580302
0.309045226130653 0.0560710458434026
0.311557788944724 0.0710345105266853
0.314070351758794 0.0855556135420418
0.316582914572864 0.0995439258951364
0.319095477386935 0.112912336502383
0.321608040201005 0.125577594668172
0.324120603015075 0.137460828521837
0.326633165829146 0.148488036185856
0.329145728643216 0.158590546616556
0.331658291457286 0.167705447247475
0.334170854271357 0.175775975772292
0.336683417085427 0.182751873627487
0.339195979899497 0.188589698973487
0.341708542713568 0.193253097225204
0.344221105527638 0.196713027447274
0.346733668341709 0.198947943204131
0.349246231155779 0.199943926738696
0.351758793969849 0.199694775644076
0.35427135678392 0.198202041488558
0.35678391959799 0.195475020153342
0.35929648241206 0.191530693943186
0.361809045226131 0.186393625830487
0.364321608040201 0.18009580649135
0.366834170854271 0.172676455086245
0.369346733668342 0.164181775025853
0.371859296482412 0.154664666243068
0.374371859296482 0.144184395762944
0.376884422110553 0.132806228622086
0.379396984924623 0.120601021435923
0.381909547738693 0.107644781144858
0.384422110552764 0.0940181916872007
0.386934673366834 0.0798061115464687
0.389447236180904 0.0650970453020426
0.391959798994975 0.049982592474062
0.394472361809045 0.0345568770948203
0.396984924623116 0.0189159615589691
0.399497487437186 0.00315724840272762
0.402010050251256 -0.0126211262625344
0.404522613065327 -0.0283209038849903
0.407035175879397 -0.0438443153695381
0.409547738693467 -0.0590946899269338
0.412060301507538 -0.0739770570833091
0.414572864321608 -0.0883987381011155
0.417085427135678 -0.102269923128465
0.419597989949749 -0.115504230482718
0.422110552763819 -0.128019244585421
0.424623115577889 -0.139737029198661
0.42713567839196 -0.150584612766675
0.42964824120603 -0.160494442840315
0.4321608040201 -0.16940480675446
0.434673366834171 -0.177260215938642
0.437185929648241 -0.184011751467635
0.439698492462312 -0.189617368700102
0.442211055276382 -0.194042159108197
0.444723618090452 -0.1972585676676
0.447236180904523 -0.199246564454201
0.449748743718593 -0.199993769378831
0.452261306532663 -0.199495529283268
0.454773869346734 -0.197754946917401
0.457286432160804 -0.194782861617108
0.459798994974874 -0.190597781803168
0.462311557788945 -0.185225769721568
0.464824120603015 -0.178700279142975
0.467336683417085 -0.171061947032082
0.469849246231156 -0.162358340484204
0.472361809045226 -0.152643660505034
0.474874371859296 -0.141978404478267
0.477386934673367 -0.130428989423038
0.479899497487437 -0.118067338387306
0.482412060301508 -0.104970432552894
0.484924623115578 -0.0912198318414035
0.487437185929648 -0.0769011670064025
0.489949748743719 -0.0621036063748342
0.492462311557789 -0.046919300558474
0.494974874371859 -0.0314428085934505
0.49748743718593 -0.0157705090814957
0.5 -1.22464679914735e-16
};
%\addlegendentry{5*w0 (M3)}
\addplot [semithick, black]
table {%
0 -0
0.00251256281407035 -0.0157633012991408
0.0050251256281407 -0.0313852777482863
0.00753768844221106 -0.0467258715361691
0.0100502512562814 -0.0616475475694245
0.0125628140703518 -0.0760165265345026
0.0150753768844221 -0.0897039842873814
0.0175879396984925 -0.10258720681803
0.0201005025125628 -0.114550690434864
0.0226130653266332 -0.125487177305563
0.0251256281407035 -0.13529861707017
0.0276381909547739 -0.143897045905204
0.0301507537688442 -0.15120537515758
0.0326633165829146 -0.157158082477906
0.0351758793969849 -0.161701799256831
0.0376884422110553 -0.164795789097834
0.0402010050251256 -0.16641231303673
0.042713567839196 -0.166536878233546
0.0452261306532663 -0.165168367907132
0.0477386934673367 -0.16231905134759
0.050251256281407 -0.158014473916752
0.0527638190954774 -0.152293228022906
0.0552763819095477 -0.145206607123077
0.0577889447236181 -0.136818145854877
0.0603015075376884 -0.127203050420861
0.0628140703517588 -0.116447524332218
0.0653266331658292 -0.10464799555681
0.0678391959798995 -0.0919102520005179
0.0703517587939698 -0.0783484930726672
0.0728643216080402 -0.0640843058386686
0.0753768844221106 -0.0492455749391117
0.0778894472361809 -0.0339653360483588
0.0804020100502513 -0.0183805831518793
0.0829145728643216 -0.00263104033560614
0.085427135678392 0.013142090901246
0.0879396984924623 0.0287973975790166
0.0904522613065327 0.0441945230664463
0.092964824120603 0.0591954254389047
0.0954773869346734 0.0736656150842631
0.0979899497487437 0.0874753604607447
0.100502512562814 0.100500851196602
0.103015075376884 0.112625308103935
0.105527638190955 0.12374003015488
0.108040201005025 0.133745369033596
0.110552763819095 0.142551622526735
0.113065326633166 0.150079838742792
0.115577889447236 0.156262523950137
0.118090452261307 0.16104424768767
0.120603015075377 0.164382139723
0.123115577889447 0.166246274402723
0.125628140703518 0.166619938948913
0.128140703517588 0.165499783296445
0.130653266331658 0.162895850127785
0.133165829145729 0.158831484835973
0.135678391959799 0.153343126223029
0.138190954773869 0.146479979810243
0.14070351758794 0.138303576689288
0.14321608040201 0.128887221869223
0.14572864321608 0.118315337065222
0.148241206030151 0.106682703821185
0.150753768844221 0.0940936137519865
0.153266331658291 0.0806609335238065
0.155778894472362 0.0665050929553904
0.158291457286432 0.0517530053123615
0.160804020100503 0.0365369294746154
0.163316582914573 0.0209932841779531
0.165829145728643 0.00526142496080171
0.168341708542714 -0.0105176052187789
0.170854271356784 -0.0262023404945416
0.173366834170854 -0.0416521603950512
0.175879396984925 -0.0567285505673522
0.178391959798995 -0.0712963446183682
0.180904522613065 -0.0852249359403876
0.183417085427136 -0.0983894486560879
0.185929648241206 -0.110671857185072
0.188442211055276 -0.121962044394513
0.190954773869347 -0.13215878884713
0.193467336683417 -0.141170672295383
0.195979899497487 -0.148916899285812
0.198492462311558 -0.155328021525406
0.201005025125628 -0.160346560515744
0.203517587939699 -0.163927522872728
0.206030150753769 -0.166038803711834
0.208542713567839 -0.166661474482359
0.21105527638191 -0.16578995267011
0.21356783919598 -0.163432051847067
0.21608040201005 -0.159608911619322
0.218592964824121 -0.154354808101307
0.221105527638191 -0.147716846615535
0.223618090452261 -0.139754539372896
0.226130653266332 -0.130539271919815
0.228643216080402 -0.120153663135786
0.231155778894472 -0.108690824519198
0.233668341708543 -0.0962535254022647
0.236180904522613 -0.0829532715792807
0.238693467336683 -0.0689093056086892
0.241206030150754 -0.0542475377517021
0.243718592964824 -0.0390994171320616
0.246231155778894 -0.023600753237492
0.248743718592965 -0.00789049832859307
0.251256281407035 0.00789049832859295
0.253768844221106 0.0236007532374919
0.256281407035176 0.0390994171320612
0.258793969849246 0.0542475377517017
0.261306532663317 0.0689093056086888
0.263819095477387 0.0829532715792804
0.266331658291457 0.0962535254022646
0.268844221105528 0.108690824519198
0.271356783919598 0.120153663135786
0.273869346733668 0.130539271919815
0.276381909547739 0.139754539372896
0.278894472361809 0.147716846615535
0.281407035175879 0.154354808101307
0.28391959798995 0.159608911619322
0.28643216080402 0.163432051847067
0.28894472361809 0.165789952670109
0.291457286432161 0.166661474482359
0.293969849246231 0.166038803711834
0.296482412060302 0.163927522872728
0.298994974874372 0.160346560515744
0.301507537688442 0.155328021525406
0.304020100502513 0.148916899285812
0.306532663316583 0.141170672295383
0.309045226130653 0.13215878884713
0.311557788944724 0.121962044394513
0.314070351758794 0.110671857185072
0.316582914572864 0.0983894486560881
0.319095477386935 0.0852249359403881
0.321608040201005 0.0712963446183686
0.324120603015075 0.0567285505673521
0.326633165829146 0.0416521603950517
0.329145728643216 0.026202340494542
0.331658291457286 0.0105176052187788
0.334170854271357 -0.00526142496080158
0.336683417085427 -0.0209932841779528
0.339195979899497 -0.0365369294746148
0.341708542713568 -0.0517530053123611
0.344221105527638 -0.0665050929553901
0.346733668341709 -0.0806609335238062
0.349246231155779 -0.0940936137519861
0.351758793969849 -0.106682703821185
0.35427135678392 -0.118315337065222
0.35678391959799 -0.128887221869223
0.35929648241206 -0.138303576689287
0.361809045226131 -0.146479979810243
0.364321608040201 -0.153343126223029
0.366834170854271 -0.158831484835973
0.369346733668342 -0.162895850127784
0.371859296482412 -0.165499783296445
0.374371859296482 -0.166619938948913
0.376884422110553 -0.166246274402723
0.379396984924623 -0.164382139723
0.381909547738693 -0.16104424768767
0.384422110552764 -0.156262523950138
0.386934673366834 -0.150079838742792
0.389447236180904 -0.142551622526735
0.391959798994975 -0.133745369033596
0.394472361809045 -0.123740030154881
0.396984924623116 -0.112625308103935
0.399497487437186 -0.100500851196603
0.402010050251256 -0.0874753604607451
0.404522613065327 -0.0736656150842633
0.407035175879397 -0.0591954254389049
0.409547738693467 -0.0441945230664468
0.412060301507538 -0.0287973975790166
0.414572864321608 -0.0131420909012464
0.417085427135678 0.00263104033560587
0.419597989949749 0.0183805831518789
0.422110552763819 0.0339653360483582
0.424623115577889 0.0492455749391112
0.42713567839196 0.0640843058386686
0.42964824120603 0.0783484930726668
0.4321608040201 0.0919102520005176
0.434673366834171 0.104647995556809
0.437185929648241 0.116447524332218
0.439698492462312 0.127203050420861
0.442211055276382 0.136818145854877
0.444723618090452 0.145206607123076
0.447236180904523 0.152293228022906
0.449748743718593 0.158014473916752
0.452261306532663 0.16231905134759
0.454773869346734 0.165168367907132
0.457286432160804 0.166536878233546
0.459798994974874 0.16641231303673
0.462311557788945 0.164795789097834
0.464824120603015 0.161701799256831
0.467336683417085 0.157158082477906
0.469849246231156 0.15120537515758
0.472361809045226 0.143897045905204
0.474874371859296 0.13529861707017
0.477386934673367 0.125487177305563
0.479899497487437 0.114550690434864
0.482412060301508 0.10258720681803
0.484924623115578 0.0897039842873817
0.487437185929648 0.0760165265345029
0.489949748743719 0.0616475475694247
0.492462311557789 0.0467258715361693
0.494974874371859 0.0313852777482865
0.49748743718593 0.015763301299141
0.5 1.22464679914735e-16
};
%\addlegendentry{6*w0 (P5)}
\addplot [semithick, black]
table {%
0 -0
0.00251256281407035 -0.0157547855587536
0.0050251256281407 -0.0313173681210988
0.00753768844221106 -0.0464978895014588
0.0100502512562814 -0.0611111525300299
0.0125628140703518 -0.074978880394924
0.0150753768844221 -0.0879318915583113
0.0175879396984925 -0.0998121637133295
0.0201005025125628 -0.110474761602191
0.0226130653266332 -0.119789605176768
0.0251256281407035 -0.127643056530696
0.0276381909547739 -0.133939306242975
0.0301507537688442 -0.138601542220141
0.0326633165829146 -0.141572886777542
0.0351758793969849 -0.14281709052764
0.0376884422110553 -0.142318974610144
0.0402010050251256 -0.140084615868915
0.042713567839196 -0.136141272716548
0.0452261306532663 -0.130537052591062
0.0477386934673367 -0.123340325061603
0.050251256281407 -0.114638887743083
0.0527638190954774 -0.104538895195297
0.0552763819095477 -0.0931635638735981
0.0577889447236181 -0.080651668930274
0.0603015075376884 -0.0671558512051433
0.0628140703517588 -0.0528407550595067
0.0653266331658292 -0.0378810197712398
0.0678391959798995 -0.0224591489953216
0.0703517587939698 -0.00676328428165117
0.0728643216080402 0.0090150901875247
0.0753768844221106 0.024683483639157
0.0778894472361809 0.0400507470310019
0.0804020100502513 0.0549294050045729
0.0829145728643216 0.0691379430204055
0.085427135678392 0.0825030217733697
0.0879396984924623 0.0948615918729187
0.0904522613065327 0.106062882989898
0.092964824120603 0.115970243203003
0.0954773869346734 0.124462806105494
0.0979899497487437 0.131436965334025
0.100502512562814 0.136807638530847
0.103015075376884 0.140509305319481
0.105527638190955 0.142496806630906
0.108040201005025 0.142745895628754
0.110552763819095 0.141253533512429
0.113065326633166 0.138037926589432
0.115577889447236 0.133138304164634
0.118090452261307 0.126614439956173
0.120603015075377 0.118545922876532
0.123115577889447 0.109031186075024
0.125628140703518 0.0981863060870263
0.128140703517588 0.086143586739945
0.130653266331658 0.073049945091761
0.133165829145729 0.0590651190931622
0.135678391959799 0.044359718839167
0.138190954773869 0.0291131451843075
0.14070351758794 0.0135114011135493
0.14321608040201 -0.00225517743051949
0.14572864321608 -0.0179942435811023
0.148241206030151 -0.0335137861131954
0.150753768844221 -0.0486244719148732
0.153266331658291 -0.0631419557865112
0.155778894472362 -0.0768891293891839
0.158291457286432 -0.0896982819058369
0.160804020100503 -0.101413146055905
0.163316582914573 -0.111890804502699
0.165829145728643 -0.121003433396043
0.168341708542714 -0.128639861779536
0.170854271356784 -0.134706927838205
0.173366834170854 -0.139130615440791
0.175879396984925 -0.141856957111238
0.178391959798995 -0.142852692413451
0.180904522613065 -0.142105673717237
0.183417085427136 -0.139625014395244
0.185929648241206 -0.13544097764293
0.188442211055276 -0.129604607277926
0.190954773869347 -0.122187105022916
0.193467336683417 -0.113278961868968
0.195979899497487 -0.102988854116388
0.198492462311558 -0.0914423175610154
0.201005025125628 -0.078780216000444
0.203517587939699 -0.0651570227438594
0.206030150753769 -0.0507389360904898
0.208542713567839 -0.0357018517671871
0.21105527638191 -0.0202292170607075
0.21356783919598 -0.00450979282354434
0.21608040201005 0.0112646493439251
0.218592964824121 0.026901666641388
0.221105527638191 0.0422104928049527
0.223618090452261 0.0570043653903344
0.226130653266332 0.0711028042108117
0.228643216080402 0.0843338131337895
0.231155778894472 0.0965359783748011
0.233668341708543 0.107560437690482
0.236180904522613 0.117272696447037
0.238693467336683 0.12555426840878
0.241206030150754 0.132304121229692
0.243718592964824 0.137439909013495
0.246231155778894 0.140898976905429
0.248743718592965 0.142639125460054
0.251256281407035 0.142639125460054
0.253768844221106 0.140898976905429
0.256281407035176 0.137439909013495
0.258793969849246 0.132304121229692
0.261306532663317 0.12555426840878
0.263819095477387 0.117272696447038
0.266331658291457 0.107560437690483
0.268844221105528 0.0965359783748014
0.271356783919598 0.0843338131337898
0.273869346733668 0.0711028042108121
0.276381909547739 0.0570043653903345
0.278894472361809 0.0422104928049531
0.281407035175879 0.0269016666413884
0.28391959798995 0.0112646493439252
0.28643216080402 -0.00450979282354421
0.28894472361809 -0.0202292170607073
0.291457286432161 -0.0357018517671865
0.293969849246231 -0.0507389360904895
0.296482412060302 -0.0651570227438592
0.298994974874372 -0.0787802160004437
0.301507537688442 -0.0914423175610151
0.304020100502513 -0.102988854116388
0.306532663316583 -0.113278961868968
0.309045226130653 -0.122187105022916
0.311557788944724 -0.129604607277926
0.314070351758794 -0.13544097764293
0.316582914572864 -0.139625014395244
0.319095477386935 -0.142105673717237
0.321608040201005 -0.142852692413451
0.324120603015075 -0.141856957111239
0.326633165829146 -0.139130615440791
0.329145728643216 -0.134706927838205
0.331658291457286 -0.128639861779536
0.334170854271357 -0.121003433396043
0.336683417085427 -0.111890804502699
0.339195979899497 -0.101413146055905
0.341708542713568 -0.0896982819058372
0.344221105527638 -0.0768891293891841
0.346733668341709 -0.0631419557865116
0.349246231155779 -0.0486244719148735
0.351758793969849 -0.0335137861131957
0.35427135678392 -0.0179942435811028
0.35678391959799 -0.00225517743051974
0.35929648241206 0.013511401113549
0.361809045226131 0.0291131451843073
0.364321608040201 0.0443597188391667
0.366834170854271 0.0590651190931618
0.369346733668342 0.0730499450917609
0.371859296482412 0.0861435867399449
0.374371859296482 0.0981863060870261
0.376884422110553 0.109031186075024
0.379396984924623 0.118545922876532
0.381909547738693 0.126614439956172
0.384422110552764 0.133138304164633
0.386934673366834 0.138037926589432
0.389447236180904 0.141253533512429
0.391959798994975 0.142745895628754
0.394472361809045 0.142496806630906
0.396984924623116 0.140509305319481
0.399497487437186 0.136807638530847
0.402010050251256 0.131436965334025
0.404522613065327 0.124462806105494
0.407035175879397 0.115970243203003
0.409547738693467 0.106062882989898
0.412060301507538 0.0948615918729188
0.414572864321608 0.0825030217733702
0.417085427135678 0.0691379430204059
0.419597989949749 0.0549294050045733
0.422110552763819 0.0400507470310022
0.424623115577889 0.0246834836391571
0.42713567839196 0.00901509018752476
0.42964824120603 -0.0067632842816508
0.4321608040201 -0.0224591489953213
0.434673366834171 -0.0378810197712392
0.437185929648241 -0.0528407550595062
0.439698492462312 -0.0671558512051429
0.442211055276382 -0.0806516689302738
0.444723618090452 -0.093163563873598
0.447236180904523 -0.104538895195297
0.449748743718593 -0.114638887743082
0.452261306532663 -0.123340325061603
0.454773869346734 -0.130537052591062
0.457286432160804 -0.136141272716548
0.459798994974874 -0.140084615868915
0.462311557788945 -0.142318974610144
0.464824120603015 -0.14281709052764
0.467336683417085 -0.141572886777542
0.469849246231156 -0.138601542220141
0.472361809045226 -0.133939306242975
0.474874371859296 -0.127643056530696
0.477386934673367 -0.119789605176768
0.479899497487437 -0.110474761602191
0.482412060301508 -0.0998121637133294
0.484924623115578 -0.0879318915583116
0.487437185929648 -0.0749788803949242
0.489949748743719 -0.0611111525300305
0.492462311557789 -0.0464978895014593
0.494974874371859 -0.0313173681210991
0.49748743718593 -0.0157547855587538
0.5 -1.22464679914735e-16
};
%\addlegendentry{7*w0 (m7)}
%\addplot [semithick, black]
%table {%
%0 -0
%0.00251256281407035 -0.0157449631334647
%0.0050251256281407 -0.0312391202962885
%0.00753768844221106 -0.0462356606770684
%0.0100502512562814 -0.0604957001428548
%0.0125628140703518 -0.073792086492066
%0.0150753768844221 -0.085913017826148
%0.0175879396984925 -0.0966654164019174
%0.0201005025125628 -0.105878004221537
%0.0226130653266332 -0.113404031368185
%0.0251256281407035 -0.11912361362698
%0.0276381909547739 -0.122945642154546
%0.0301507537688442 -0.124809234777547
%0.0326633165829146 -0.124684705802042
%0.0351758793969849 -0.122574038885301
%0.0376884422110553 -0.118510855437564
%0.0402010050251256 -0.112559879057094
%0.042713567839196 -0.104815904529672
%0.0452261306532663 -0.0954022878156461
%0.0477386934673367 -0.0844689810779511
%0.050251256281407 -0.0721901440516986
%0.0527638190954774 -0.0587613698045004
%0.0552763819095477 -0.0443965690791786
%0.0577889447236181 -0.0293245628490462
%0.0603015075376884 -0.0137854373639094
%0.0628140703517588 0.00197328025170463
%0.0653266331658292 0.0177005649281189
%0.0678391959798995 0.0331458922998348
%0.0703517587939698 0.0480632293790014
%0.0728643216080402 0.0622149536844603
%0.0753768844221106 0.0753756383974518
%0.0778894472361809 0.0873356432491633
%0.0804020100502513 0.0979044539398613
%0.0829145728643216 0.106913716895051
%0.085427135678392 0.11421992101718
%0.0879396984924623 0.119706683714491
%0.0904522613065327 0.12328660479225
%0.092964824120603 0.124902658675159
%0.0954773869346734 0.124529102783876
%0.0979899497487437 0.122171887595838
%0.100502512562814 0.11786856185843
%0.103015075376884 0.111687674464359
%0.105527638190955 0.103727682516966
%0.108040201005025 0.0941153829791722
%0.110552763819095 0.083003892888804
%0.113065326633166 0.0705702103139898
%0.115577889447236 0.0570123949008771
%0.118090452261307 0.0425464129255142
%0.120603015075377 0.0274026971059614
%0.123115577889447 0.0118224759743557
%0.125628140703518 -0.00394606872060133
%0.128140703517588 -0.0196517553709062
%0.130653266331658 -0.0350444036521267
%0.133165829145729 -0.0498788197165425
%0.135678391959799 -0.0639187019552908
%0.138190954773869 -0.0769404051135224
%0.14070351758794 -0.0887365027989166
%0.14321608040201 -0.0991190916353472
%0.14572864321608 -0.107922784428903
%0.148241206030151 -0.115007344667272
%0.150753768844221 -0.120259920386808
%0.153266331658291 -0.123596841823375
%0.155778894472362 -0.124964954211685
%0.158291457286432 -0.124342464502582
%0.160804020100503 -0.121739288510692
%0.163316582914573 -0.117196892962603
%0.165829145728643 -0.110787634961651
%0.168341708542714 -0.102613609391158
%0.170854271356784 -0.0928050226161605
%0.173366834170854 -0.0815181183893985
%0.175879396984925 -0.0689326890003885
%0.178391959798995 -0.0552492113131975
%0.180904522613065 -0.0406856533137766
%0.183417085427136 -0.0254740020362692
%0.185929648241206 -0.00985656817593455
%0.188442211055276 0.00591787374644471
%0.190954773869347 0.0215980481842624
%0.193467336683417 0.0369341812043336
%0.195979899497487 0.0516819792065166
%0.198492462311558 0.0656065203455584
%0.201005025125628 0.0784859966676071
%0.203517587939699 0.0901152473518395
%0.206030150753769 0.100309026775197
%0.208542713567839 0.108904955342307
%0.21105527638191 0.11576610607598
%0.21356783919598 0.120783185765753
%0.21608040201005 0.123876275930349
%0.218592964824121 0.12499610586177
%0.221105527638191 0.124124837472334
%0.223618090452261 0.121276349442623
%0.226130653266332 0.116496016144054
%0.228643216080402 0.109859984857683
%0.231155778894472 0.101473962802628
%0.233668341708543 0.0914715332958847
%0.236180904522613 0.0800120278658885
%0.238693467336683 0.0672779882155362
%0.241206030150754 0.0534722584637762
%0.243718592964824 0.0388147539842712
%0.246231155778894 0.0235389583112148
%0.248743718592965 0.00788820391408435
%0.251256281407035 -0.00788820391408424
%0.253768844221106 -0.0235389583112145
%0.256281407035176 -0.0388147539842708
%0.258793969849246 -0.0534722584637759
%0.261306532663317 -0.067277988215536
%0.263819095477387 -0.0800120278658883
%0.266331658291457 -0.0914715332958844
%0.268844221105528 -0.101473962802627
%0.271356783919598 -0.109859984857683
%0.273869346733668 -0.116496016144054
%0.276381909547739 -0.121276349442623
%0.278894472361809 -0.124124837472334
%0.281407035175879 -0.12499610586177
%0.28391959798995 -0.123876275930349
%0.28643216080402 -0.120783185765753
%0.28894472361809 -0.11576610607598
%0.291457286432161 -0.108904955342308
%0.293969849246231 -0.100309026775197
%0.296482412060302 -0.0901152473518396
%0.298994974874372 -0.0784859966676076
%0.301507537688442 -0.0656065203455587
%0.304020100502513 -0.0516819792065167
%0.306532663316583 -0.036934181204334
%0.309045226130653 -0.0215980481842625
%0.311557788944724 -0.00591787374644506
%0.314070351758794 0.00985656817593443
%0.316582914572864 0.0254740020362688
%0.319095477386935 0.0406856533137763
%0.321608040201005 0.0552492113131974
%0.324120603015075 0.0689326890003885
%0.326633165829146 0.0815181183893982
%0.329145728643216 0.0928050226161601
%0.331658291457286 0.102613609391158
%0.334170854271357 0.110787634961651
%0.336683417085427 0.117196892962603
%0.339195979899497 0.121739288510692
%0.341708542713568 0.124342464502582
%0.344221105527638 0.124964954211685
%0.346733668341709 0.123596841823375
%0.349246231155779 0.120259920386808
%0.351758793969849 0.115007344667272
%0.35427135678392 0.107922784428903
%0.35678391959799 0.0991190916353474
%0.35929648241206 0.0887365027989169
%0.361809045226131 0.0769404051135226
%0.364321608040201 0.0639187019552909
%0.366834170854271 0.0498788197165429
%0.369346733668342 0.035044403652127
%0.371859296482412 0.0196517553709063
%0.374371859296482 0.00394606872060156
%0.376884422110553 -0.0118224759743556
%0.379396984924623 -0.0274026971059615
%0.381909547738693 -0.042546412925514
%0.384422110552764 -0.057012394900877
%0.386934673366834 -0.0705702103139896
%0.389447236180904 -0.0830038928888035
%0.391959798994975 -0.0941153829791719
%0.394472361809045 -0.103727682516965
%0.396984924623116 -0.111687674464359
%0.399497487437186 -0.11786856185843
%0.402010050251256 -0.122171887595838
%0.404522613065327 -0.124529102783876
%0.407035175879397 -0.124902658675159
%0.409547738693467 -0.12328660479225
%0.412060301507538 -0.119706683714491
%0.414572864321608 -0.11421992101718
%0.417085427135678 -0.106913716895052
%0.419597989949749 -0.0979044539398614
%0.422110552763819 -0.0873356432491636
%0.424623115577889 -0.075375638397452
%0.42713567839196 -0.0622149536844604
%0.42964824120603 -0.0480632293790016
%0.4321608040201 -0.0331458922998349
%0.434673366834171 -0.0177005649281193
%0.437185929648241 -0.00197328025170481
%0.439698492462312 0.0137854373639094
%0.442211055276382 0.0293245628490458
%0.444723618090452 0.0443965690791785
%0.447236180904523 0.0587613698045001
%0.449748743718593 0.0721901440516984
%0.452261306532663 0.0844689810779508
%0.454773869346734 0.0954022878156459
%0.457286432160804 0.104815904529672
%0.459798994974874 0.112559879057094
%0.462311557788945 0.118510855437563
%0.464824120603015 0.1225740388853
%0.467336683417085 0.124684705802042
%0.469849246231156 0.124809234777547
%0.472361809045226 0.122945642154546
%0.474874371859296 0.11912361362698
%0.477386934673367 0.113404031368185
%0.479899497487437 0.105878004221537
%0.482412060301508 0.0966654164019174
%0.484924623115578 0.0859130178261483
%0.487437185929648 0.0737920864920662
%0.489949748743719 0.0604957001428552
%0.492462311557789 0.0462356606770686
%0.494974874371859 0.0312391202962886
%0.49748743718593 0.015744963133465
%0.5 1.22464679914735e-16
%};
%\addlegendentry{8*w0 (3x octave)}
\end{axis}

% Annotations
%\draw[-latex,black, line width=1 pt] (0.25,1.5) -- (0.5,1.5) ; 
%\node[black] (note) at (0.35,1.5) {$n=1$};

\end{tikzpicture}
\end{document}

}
	\caption{The harmonic series}
	\label{fig:serie_harmonique}
\end{figure} 

%% Table
\input{tableaux/harmoniques}


% Table
% Source: https://hellomusictheory.com/learn/intervals/
\begin{table*}[!h]
	\caption{Intervals chart in relation to C note. Minor (m or ``-''), major (M or ``maj''), augmented (A or ``aug'' or ``$\#$'' or ``$+$'') and diminished (d or ``dim'' or ``b''). }
	\centering
	\begin{adjustwidth}{-2cm}{}
	\begin{tabular}{clclcl}
		\hline 
		\textbf{Semitones} & \textbf{Name} & \textbf{Notation} & \textbf{Songs}  \\
		\hline
		0 & Perfect unison            & P1 & -   \\
		\hline
		1 & Minor second              & m2 & JAWS theme \\
		2 & Major second              & M2 & \textbf{Frè-re} Jacques  \\
		\hline
		3 & Minor third               & m3 & Iron Man by Black Sabbath\\
		4 & Major third               & M3 & "\textbf{Oh-When} the Saints" \\
		\hline
		5 & Perfect fourth            & P4 & Here Comes the Bride (Wedding song) \\
		\hline
		6 & Triton                    & T  & "\textbf{The - Simp}-sons" \\ 
		\hline
		7 & Perfect fifth             & P5 &  "\textbf{Twinkle - Twinkle} Little Star"   \\
		\hline
		8 & Minor sixth               & m6 &  The Entertainer \\
		9 & Major sixth               & M6 &  Jingle Bells ("\textbf{Dash-ing} through the snow")   \\
		\hline
		10 & Minor seventh            & m7 &  Theme song Star Trek : The Original Series\\
		11 & Major seventh            & M7 &  Take On Me ("Take-on")  \\
		\hline
		12 & Perfect octave           & P8 &  "\textbf{Some-where} over the rainbow" \\
		\hline
		13 & Minor ninth              & m9 &  - \\
		14 & Major ninth              & M9 &  - \\
		\hline
		16 & Diminished eleventh      & d11 & - \\
		17 & Perfect eleventh         & P11 & - \\
		18 & Augmented eleventh       & A11 & - \\
		\hline
		20 & Minor thirteenth         & m13 & - \\
		21 & Major thirteenth         & M13 & - \\
		\hline
	\end{tabular}
	\end{adjustwidth}
	\label{tab: }
\end{table*}


% ****************************************************************************
\subsection{Consonance and dissonance}

% Figure 
\begin{figure}[h!]
	\centering
	\hspace*{0cm}
	\includegraphics[scale=0.03, trim= {0cm 0cm 0cm 0cm}, clip]{Harmonic_entropy.png}
	\caption{Harmonic entropy}
	\label{fig}
\end{figure}

% Figure 
\begin{figure}[h!]
	\centering
	\hspace*{0cm}
	\includegraphics[scale=0.5, trim= {0cm 0cm 0cm 0cm}, clip]{battement/main.pdf}
%	\scalebox{0.3}{\documentclass{standalone}
\usepackage{pgfplots} % Include package for TikZ and PGF plot
\usepackage{anyfontsize} % enable to change the font size manually 
\usepackage{makecell}%
\usetikzlibrary{shapes.geometric}
\tikzset{
dot/.style = {circle, fill, minimum size=#1,
              inner sep=0pt, outer sep=0pt},
dot/.default = 6pt % size of the circle diameter 
}
 \renewcommand{\familydefault}{\sfdefault}

\begin{document}
\begin{tikzpicture}[scale=1]
	\def \R{10cm}
	
	
	\draw[black, thick] (0,0) circle (\R); 
	\draw[black, thick] (0,0) circle (2*\R/3); 
	
	
	\node[] at (0,8cm) {{\large C}};
	
	%\fill[black, line width=2] (0.0,-0.4) rectangle (20*\fret,4.4);
	%\draw[color=black!50!white, thick]  (0, 0)   -- (0,5*\h); 
	
	
\end{tikzpicture}
\end{document}}
	\caption{Beat tone}
	\label{fig}
\end{figure}


%%%%%%%%%%%%%%%%%%%%%%%%%%%%%%%%%%%%%%%%%%%%%%%%%%%%%%%%%%%%%%%%%%%%%%%
% Interval of notes create scales
\newpage
\section{Scales}
%%%%%%%%%%%%%%%%%%%%%%%%%%%%%%%%%%%%%%%%%%%%%%%%%%%%%%%%%%%%%%%%%%%%%%%

% Table
\input{tableaux/gammes.tex}

\newpage
\subsection{Major scale}
Modes ranked by brightness: Super-locrian, locrian, phrygian, aeolian, dorian, mixolydian, major, lydian, lydian augmented

\begin{itemize}
	\item Major scales and the modes (and all modes)
	\item Pentatonic scale (Major, Egyptian, Man Gong, Ritusen)
	\item Minor scale (natural, harmonic, melodic)
	\item Phrygian dominant (hijaz) (I-bII-iiidim-iv-vdim-bVI+-bvii)  Ex: Come out and Play The Offsprings
\end{itemize}

% Figure 
\begin{figure}[h!]
	\centering
	\hspace*{-2cm}
	\includegraphics[scale=0.55, trim= {0cm 0cm 0cm 0cm}, clip]{gamme_majeure_manche/main.pdf}
	\caption{G Major scale on the fretboard}
	\label{fig:gamme_majeure_manche}
\end{figure}

\newpage
\subsection{Pentatonic scale}
% Figure 
\begin{figure}[h!]
	\centering
	\hspace*{-2cm}
	\includegraphics[scale=0.55, trim= {0cm 0cm 0cm 0cm}, clip]{gamme_penta_manche/main.pdf}
	\caption{Pattern of pentatonic scales}
	\label{fig:gammme_penta_manche}
\end{figure}

\newpage
\subsection{Blues}


% Table
\begin{table*}[!h]
	\caption{Blues scales (relative to the major scale)}
	\centering
	%\begin{adjustwidth}{0cm}{}
	\begin{tabular}{l|cccccccc}
		Scale name  & \multicolumn{8}{c}{Formula} \\
		\hline \hline \vspace{-0.4cm} \\
		Blues Major   & 1 & 2  & \textcolor{red}{b3} & 3  &   -   & 5  & 6  &  -  \\
		Blues minor   & 1 &  - & b3 & 4  & \textcolor{red}{b5} &  5  & - &  b7 \\
	\end{tabular}
	\label{tab: }
	%\end{adjustwidth}
\end{table*}

% Table
\begin{table*}[!h]
	\caption{12 bar blues in C major}
	\centering
	\begin{tabular}{| c | c | c | c |}
		\hline
		\phantom{x}G7\phantom{x} & \phantom{x}G7\phantom{x} & \phantom{x}G7\phantom{x} & \phantom{x}G7\phantom{x}  \\
		\hline
		\phantom{x}C7\phantom{x} & \phantom{x}C7\phantom{x} & \phantom{x}G7\phantom{x} & \phantom{x}G7\phantom{x}  \\
		\hline
		\phantom{x}D7\phantom{x} & \phantom{x}C7\phantom{x} & \phantom{x}G7\phantom{x} & \phantom{x}D7\phantom{x}  \\
		\hline
	\end{tabular}
	\label{tab: }
\end{table*}

% Figure
\begin{figure}[h!]
	\centering
	\hspace*{-1cm}
	\scalebox{0.7}{\input{figures/blues/penta_mineur.tex}}
	\caption{ }
	\label{fig:blues_penta_mineur}
\end{figure}

%%%%%%%%%%%%%%%%%%%%%%%%%%%%%%%%%%%%%%%%%%%%%%%%%%%%%%%%%%%%%%%%%%
% Construction of chords superimposing 3rds on a scale
\newpage
\section{Chords}
%%%%%%%%%%%%%%%%%%%%%%%%%%%%%%%%%%%%%%%%%%%%%%%%%%%%%%%%%%%%%%%%%%%%%%%

\begin{itemize}
	\item Tonic: I, iii, vi 
	\item Pre-dominant: IV, ii
	\item Dominant: V, vii$^\circ$
\end{itemize}


\begin{itemize}
	\item ``Sus'' chords: chord without third
	\item ``sus9'' will often replace the dominant 7th chord
\end{itemize}

%%%%%%%%%%%%%%%%%%%%%%%%%%%%%%%%%%%%%%%%%%%%%%%%%%%%%%%%%%%%%%%%%%%%%%%%%%%%%%%%%%%%%%%%%%%%%%%%%%%%%%%%%
\newpage
\subsection{Formation of chords}

% Table
% Source: 1997 - Vaillot Méthode Jazz
\begin{table*}[!h]
	\centering
	\caption{Construction of chords (notation is relative to the major scale)}
	\begin{tabular}{clcccccc}
		\hline \vspace{-0.2cm} \\
		$\#$ notes & Chords & & & & & &\\
		\hline \vspace{-0.2cm} \\
		\multirow{6}{*}{triad} & -        & 1 & M3 & 5 &  -  & -  & -\\
		                       & m        & 1 & m3 & 5 &  -  & -  & -\\
		                       & dim or $^\circ$  & 1 & m3 & b5  &  -  & -  & -\\
		                       & aug or $^\#$5 & 1 & m3 & $^\#$5 &  -  & -  & -\\
		                       & sus2     & 1 & M2  & 5     &  -  & -  & -\\
		                       & sus4     & 1 & 4   & 5     &  -  & -  & -\\
		\hline
		\multirow{13}{*}{tetrad}& 7        & 1 & M3  & 5 & m7  & -  & -\\
		                       & $\Delta$  & 1 & M3  & 5 & M7  & -  & -\\
		                       & m$^7$     & 1 & m3  & 5 & m7  & -  & -\\
		                       & m$^\Delta$& 1 & m3  & 5 & M7  & -  & -\\
		                       & m$^{7\textrm{b}5}$ or $\varnothing$& 1 & m3  & b5 & m7  & -  & -\\
		                       & $^{\circ 7}$   & 1 & m3  & b5 & b7 & -  & -\\
                               & 6        & 1 & M3  & 5 & 6   & -  & -\\
                               & m6       & 1 & m3  & 5 & 6   & -  & -\\
		                       & m6(9)    & 1 & m3  & 6 & M9  & -  & -\\
		                       & 6(9)     & 1 & M3  & 6 & M9  & -  & -\\
		                       & 7sus4    & 1 & 4   & 5 & m7  & -  & -\\
		                       & add2     & 1 & M2  & M3& 5   & -  & -\\
		                       & add9     & 1 & M3  & 5 & M9  & -  & -\\
		\hline
		\multirow{6}{*}{pentad}& 7(b9)      & 1 & M3  & 5 & m7  & m9  & -\\
	                           & $\Delta^9$ & 1 & M3  & 5 & M7  & M9  & -\\
		                       & 9          & 1 & M3  & 5 & m7  & M9  & -\\
		                       & m9         & 1 & m3  & 5 & m7  & M9  & -\\
		                       & sus9       & 1 & 4   & 5 & m7  & M9  & -\\
		                       & 11         & 1 & 5   & m7& M9  & 11  & -\\ % The 3 is omitted to avoid the dissonace between 3-11.
		\hline
		\multirow{2}{*}{hexad} & 7(13)      & 1 & M3  & 5 & m7  & M9  & M13\\ % the 11 is omited to avoid the dissonance between 3-11.
		                       & 7(b9,13)   & 1 & M3  & 5 & m7  & m9  & M13\\
		\hline \vspace{-0.2cm}
	\end{tabular}
	\label{tab: }
\end{table*}


%%%%%%%%%%%%%%%%%%%%%%%%%%%%%%%%%%%%%%%%%%%%%%%%%%%%%%%%%%%%%%%%%%%%%%%%%%%%%%%%%%%%%%%%%%%%%%%%%%%%%%%%%
\newpage
\subsection{Chord inversions}

% Figure 
\begin{figure}[h!]
	\centering
	\hspace*{-2.2cm}
	\includegraphics[scale=0.7, trim= {0cm 0cm 0cm 0cm}, clip]{figures/chord-inversions/maj7.pdf}
	\hspace*{-2.2cm}
	\includegraphics[scale=0.7, trim= {0cm 0cm 0cm 0cm}, clip]{figures/chord-inversions/Dominant7.pdf}
	\hspace*{-2.2cm}
	\includegraphics[scale=0.7, trim= {0cm 0cm 0cm 0cm}, clip]{figures/chord-inversions/m7.pdf}
	\caption{ }
	\label{fig}
\end{figure}

%%%%%%%%%%%%%%%%%%%%%%%%%%%%%%%%%%%%%%%%%%%%%%%%%%%%%%%%%%%%%%%%%%%%%%%%%%%%%%%%%%%%%%%%%%%%%%%%%%%%%%%%%
\newpage
\subsection{Chord progression and example}

% Table
% Source: 1997 - Vaillot Méthode Jazz

\begin{table*}[!h]
	\centering
	\caption{Famous chord progressions}
	\begin{adjustwidth}{-3cm}{}
	\begin{tabular}{p{5.0cm}p{5.5cm}p{7cm}}
		\hline \vspace{-0.2cm} \\
	    Name & Progression & Example\\
		\hline \vspace{-0.2cm} \\
		Pop major (punk)      & $\textrm{I}-\textrm{V}-\textrm{vi}-\textrm{IV}$ & Dammit, Let it be, Country Road \\
		Anatol (turnaround)   & $\textrm{I}^{\Delta}-\textrm{vi}^7-\textrm{ii}^7-\textrm{V}^7$ & Blue Moon \\
		50s progression       & $\textrm{I}-\textrm{vi}-\textrm{IV}-\textrm{V}$ & Every Breath You Take, Crocodile Rock\\
		Ragtime               &$\textrm{I}-\textrm{VI}^7-\textrm{II}^{7}-\textrm{V}^{7}$ & I want to be like you (Disney) \\
		Jazz (ii-V-I)         & $\textrm{ii}^{7}-\textrm{V}^7-\textrm{I}^{\Delta}$ & Autumn leaves\\
		Blues/Rock (Major)    & $\textrm{I}^7-\textrm{IV}^7-\textrm{V}^7-\textrm{I}^7$ & Johnny B. Goode\\

	    Mixo vamp (mixo)      & $\textrm{I}-\textrm{bVII}-\textrm{IV}-\textrm{I}$ & Hey Jude, Sweet home Alabama \\
	    Japanese ``Royal road'' & $\textrm{IV}^\Delta-\textrm{V}^7-\textrm{iii}^7-\textrm{vi}^7-$ {\footnotesize $(\textrm{ii}^7-\textrm{V}^7-\textrm{I}^{\Delta})$} & Shogo theme, anime \\
	    ``Storyteller''       & $\textrm{I}-\textrm{IV}-\textrm{vi}-\textrm{V}$ & \\
	    Creep chord            & I$\,-\,$III$\,-\,$IV$\,-\,$iv &  Creep, Space Oddity \\
		Pop minor             & $\textrm{i}-\textrm{bVI}-\textrm{bIII}-\textrm{bVII}$ & Save Tonight, Africa Toto \\
	    Aeolian vamp          & $\textrm{i}-\textrm{bVII}-\textrm{bVI}-\textrm{bVII}$  & Stairway to Heaven, All Iron Maiden \vspace{-0.8cm}  \\
		Minor progression 01    & $\textrm{i}-\textrm{i}-\textrm{bVI}-\textrm{V}$ & Sweet Dreams \\
	    Minor progression 02    & $\textrm{i}-\textrm{bVI}-\textrm{bIII}-\textrm{bVII}$ &  \\
	    Minor progression 03    & $\textrm{i}-\textrm{bVI}-\textrm{iv}-\textrm{bVII}$ & Final countdown\\
	    Minor progression 04    & $\textrm{i}-\textrm{bIII}-\textrm{bVII}-\textrm{iv}$ & Boulevard of Broken Dreams\\
	    Andalusian	(phrygian)& $\textrm{i}-\textrm{bVII}-\textrm{bVI}-\textrm{V}^7$ & Happy Together The Turtles\\
	    Blues/Rock (minor)      & $\textrm{i}^7-\textrm{iv}^7-\textrm{V}^7-\textrm{i}^7$ & Minor swing\\
	    Anime                   & $\textrm{bVI}-\textrm{bVII}-\textrm{i}$   &             \\
		Neapolitan              & i$\,-\,$bII$^6\,-\,$V$\,-\,$i &  Classic \\
		\hline \vspace{-0.2cm} \\
	\end{tabular}
	\end{adjustwidth}
	\label{tab:}
\end{table*}

Concepts: 
\begin{itemize}
	\item Borrowed chord: chord that is not built from the scale of the tonic. Examples:
	\begin{itemize}
		\item ``Picardy third'': a progression with an ending major triad instead of an expected minor triad to create an impression of resolution.
		\item Use the bVII
	\end{itemize}
	\item Transistion Chords:
	\begin{itemize}
		\item Secondary dominant chord (tonicization) (V/x): using the fifth of a chord (even if it's not a diatonic chord) in order to feel a ``resolution'' on this chord.
		\item Tritone substitution (Vsub/x or bV7/V): Approach any target chord with a diminished 7 chord a semitone above.
		\item Backdoor [ii V]. Approach the tonic with iv7 - bVII7 - I.
		\item Modulation (Rick Beato):
		\begin{itemize}
			\item Diatonic common chord (``close'' keys have many chords in common that can be used to modulate from a key to another. Common chords are called pivot chords)
			\item Chromatic pivot chord
			\item Enharmonic dominant
			\item Deceptive
			\item Enharmonic Dim7
			\item Dim7 to Dom7 (lower the root of the dim7 chord to create a dominant chord that leads to a new tonic)
			\item Chromatic Mediant
			\item Common tone (Pivot note)
			\item Direct or Linear (Abrupt change of key without preparation to ``lift'' the song)
			\item Chain Modulation ()
			\item Parallel modulation (Modulation of the mode but keep the same root ex: C to Cm)
		\end{itemize}
	\end{itemize}
\end{itemize}


% Figure 
\begin{figure}[h!]
	\centering
	\hspace*{-1cm}
	\includegraphics[scale=0.3, trim= {0cm 0cm 0cm 0cm}, clip]{Circle_5th.png}
	\caption{ }
	\label{fig}
\end{figure}

\begin{itemize}
	\item Substitution tritonique
	\item Substitution diatonique
\end{itemize}

%%%%%%%%%%%%%%%%%%%%%%%%%%%%%%%%%%%%%%%%%%%%%%%%%%%%%%%%%%%%%%%%%%
\newpage
\section{Modes}

% Table

\begin{table*}[!h]
	\caption{Table of modes }
	\begin{adjustwidth}{-2.3cm}{}
	\begin{tabular}{lccc cccc}
		\hline \vspace{-0.2cm} \\
		Mode name & Ionian & Dorian & Phrygian & Lydian & Mixolydian & Aeolian & Locrian\\
		\hline \vspace{-0.2cm} \\
		Diatonic chords & I & ii$ & iii  & IV & V & vi & vii$^{\circ}$ \\
		Diatonic seventh & $\Delta$7 & $^{-}$7 & $^{-}$7  & $\Delta$7 & 7 & $^{-}$7 & $\varnothing$ \\
		Alternative naming & maj7 & m7 & m7 & maj7 & 7 & m7 & m7b5\\
		%Diatonic seventh  & $\Delta$7 & $^{-}$7 & $^{-}$7 & $\Delta$7 & 7 & $^{-}$7 & $\varnothing$ \\
		\hline \vspace{-0.2cm} \\
		{\scriptsize $\# \# \# \# \# \#$} & F$^\#$ & G$^\#$ & A$^\#$ & B & C$^\#$ & D$^\#$ & E$^\#$ \\
		{\scriptsize $\# \# \# \# \#$}    & B & C$^\#$ & D$^\#$ & E & F$^\#$ & G$^\#$ & A$^\#$\\
		{\scriptsize $\# \# \# \#$}       & E & F$^\#$ & G$^\#$ & A & B & C$^\#$ & D$^\#$\\
		{\scriptsize $\# \# \#$}          & A & B & C$^\#$ & D & E & F$^\#$ & G$^\#$\\
		{\scriptsize $\# \#$}             & D & E & F$^\#$ & G & A & B & C$^\#$\\
		{\scriptsize $\#$}                & G & A & B & C  & D & E & F$^\#$\\
		{\scriptsize -}                   & \textbf{C} & \textbf{D} & \textbf{E} & \textbf{F} & \textbf{G} & \textbf{A} & \textbf{B}\\
		{\scriptsize b}                   & F  & G  & A  & B$^\textrm{b}$ & C  & D  & E\\
		{\scriptsize bb}                  & B$^\textrm{b}$ & C     & D     & E$^\textrm{b}$ & F     & G     & A\\
		{\scriptsize bbb}                 & E$^\textrm{b}$ & F     & G     & A$^\textrm{b}$ & B$^\textrm{b}$ & C     & D\\
		{\scriptsize bbbb}                & A$^\textrm{b}$ & B$^\textrm{b}$ & C     & D$^\textrm{b}$ & E$^\textrm{b}$ & F     & G\\
		{\scriptsize bbbbb}               & D$^\textrm{b}$ & E$^\textrm{b}$ & F     & G$^\textrm{b}$ & A$^\textrm{b}$ & B$^\textrm{b}$ & C\\
		{\scriptsize bbbbbb}              & G$^\textrm{b}$ & A$^\textrm{b}$ & B$^\textrm{b}$ & C$^\textrm{b}$ & D$^\textrm{b}$ & E$^\textrm{b}$ & F\\
		\hline
		\hline \vspace{-0.2cm}
	\end{tabular}
	\label{tab: }
	\end{adjustwidth}
\end{table*}

% Table

\begin{table*}[!h]
	\caption{Harmonization of scales (relative to major scale)}
	\begin{adjustwidth}{0cm}{}
	\begin{tabular}{r|ccc cccc}
		\hline \vspace{-0.4cm} \\
		& 1 & 2 & 3 & 4 & 5 & 6 & 7\\
		\hline \vspace{-0.2cm} \\
		Major  & I$^\Delta$ & ii$^{-7}$ & iii$^{-7 \phantom{\textrm{aug}}$ & IV$^\Delta$ & V$^7$ & vi$^{-7}$ & vii$^{\varnothing}$ \\
		\hline \vspace{-0.2cm} \\
		Natural minor  & i$^{-7}$ & ii$^{\varnothing}$ & bIII$^{\Delta \phantom{\textrm{aug}}}$ & iv$^{-7}$ & v$^{-7}$ & bVI$^{\Delta}$ & bVII$^{7}$ \\
		\hline \vspace{-0.2cm} \\
		Harmonic minor & i$^{\Delta}$ & ii$^{\varnothing}$ & bIII$^{\Delta, \textrm{aug}}$ & iv$^{-7}$ & V$^7$ & bVI$^{\Delta}$ & vii$^{\circ 7}$ \\
		\hline \vspace{-0.2cm} \\
		Melodic minor & i$^{\Delta}$ & ii$^{-7}$ & bIII$^{\Delta, \textrm{aug}}$ & IV$^7$ & V$^7$ & vi$^{\varnothing}$ & vii$^{\varnothing}$ \\
		\hline \vspace{-0.2cm} \\
		Dorian        & i$^{-7}$ & ii$^{-7}$ & bIII$^{\Delta, \phantom{\textrm{aug}}}$ & IV$^{7}$ & v$^{-7}$ & vi$^{\varnothing}$ & bVII$^{\Delta}$ \\
		\hline
	\end{tabular}
	\label{tab: }
	\end{adjustwidth}
\end{table*}


\begin{itemize}
	\item Ionian (Joy), dorian(Jazz), phrygian(flamenco,doom), lydian (floaty,mystery) (ex: E.T., Jurassic Park, Back to the Future), mixo(blues)(ex: AC/DC), aeolian(sad)(ex: Losing my Religion), locrian(tension)(ex:Bjork Army of Me) 
	\item Harmonization of harmonic minor scale
	\item Harmonization of melodic minor scale
	\item How to use this table
	\item Example of chord progression
\end{itemize}


%%%%%%%%%%%%%%%%%%%%%%%%%%%%%%%%%%%%%%%%%%%%%%%%%%%%%%%%%%%%%%%%%%
\newpage
\section{Arpeggios}

% Figure
\begin{table}[!h]
	\hspace*{-4cm}
	\includegraphics[width=10.5cm, trim= {0cm 0cm 0cm 0cm}, clip]{Arpeges/maj7_chords.pdf}
	\hspace*{-1cm}
	\includegraphics[width=10.5cm, trim= {0cm 0cm 0cm 0cm}, clip]{Arpeges/2notes_maj7_chords.pdf}
	
	\hspace*{-4cm}
	\includegraphics[width=10.5cm, trim= {0cm 0cm 0cm 0cm}, clip]{Arpeges/dom7_chords.pdf}
	\hspace*{-1cm}
	\includegraphics[width=10.5cm, trim= {0cm 0cm 0cm 0cm}, clip]{Arpeges/2notes_dom7_chords.pdf}
	
	\hspace*{-4cm}
	\includegraphics[width=10.5cm, trim= {0cm 0cm 0cm 0cm}, clip]{Arpeges/m7_chords.pdf}
	\hspace*{-1cm}
	\includegraphics[width=10.5cm, trim= {0cm 0cm 0cm 0cm}, clip]{Arpeges/2notes_m7_chords.pdf}
	
	\hspace*{-4cm}
	\includegraphics[width=10.5cm, trim= {0cm 0cm 0cm 0cm}, clip]{Arpeges/m7b5_chords.pdf}
	\hspace*{-1cm}
	\includegraphics[width=10.5cm, trim= {0cm 0cm 0cm 0cm}, clip]{Arpeges/2notes_m7b5_chords.pdf}
	\caption{G arpeggio}
	\label{fig}
\end{table}


%%%%%%%%%%%%%%%%%%%%%%%%%%%%%%%%%%%%%%%%%%%%%%%%%%%%%%%%%%%%%%%%%%
\newpage
\section{Transposition}

https://www.youtube.com/watch?v=Vxac3hHrxg8




%%%%%%%%%%%%%%%%%%%%%%%%%%%%%%%%%%%%%%%%%%%%%%%%%%%%%%%%%%%%%%%%%%
\newpage
\section{Composition variation (Shred Master Scott)}

\begin{itemize}
	\item Pedal tone
	\item Inversion
	\item Voice leading
\end{itemize}

\newpage
\bibliographystyle{plain}
\bibliography{/home/jh/Documents/library.bib}

\end{document}
